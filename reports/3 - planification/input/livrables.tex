\section{Organisation}
	\label{sec:livrables}

	Au cours de nos réunions, nous nous sommes naturellement dirigés vers une méthode agile ressemblant à \og Scrum \fg. Nous nous réunissons une fois par semaine afin de discuter de l'avancement de nos tâches respectives et faire part de nos difficultés. Nous définissons en fonction de cela des tâches à effectuer la semaine suivante. Ainsi, nous avons décidé de livrer trois versions de \glasir{}, dont 2 intermédiaires.
	Chaque version sera composée de sprints au cours desquels nous nous consacrerons à une partie des tâches composant une version.
	Nous présentons ci-dessous la répartition des livrables :
	
	\begin{itemize}
		\item{v 0.1} Paramètres de synthèse :
		\begin{itemize}
			\item{sprint 1} Implémenter l'IHM, y intégrer ADTool, créer un projet, l'afficher dans l'arborescence ;
			\item{sprint 2} Évaluer une formule dans ADTool, la propager dans l'arbre, améliorer le codage des arbres ;
			\item{sprint 3} Rendre visibles plusieurs paramètres, communiquer les formules entre ADTool et \glasir{}, documenter le code et tester ;
		\end{itemize}
		\item{v 0.2} filtre :
		\begin{itemize}
			\item{sprint 1} Définir l'algorithme de filtrage, créer l'IHM ;
			\item{sprint 2} Créer un panneau dans \glasir{}, afficher l'arbre résultat, tester ;
			\item{sprint 3} Documenter le code, ouverture simultanée de plusieurs arbres ;
		\end{itemize}
		\item{v 1.0} optimiseur :
		\begin{itemize}
			\item{sprint 1} Implémenter l'algorithme de calcul du chemin optimal, créer le panneau associé dans l'IHM ;
			\item{sprint 2} Annuler une action, créer une bibliothèque de modèles, créer la page HTML ;
			\item{sprint 3} Documenter le code, harmoniser les graphismes.
		\end{itemize}
	\end{itemize}


	Ce plan d'action est amené à changer au fur et à mesure de sa réalisation, ces évolutions étant le but même des méthodes dites \og agiles \fg.
	
	\paragraph{Version intermédiaire 1}


	\paragraph{Version intermédiaire 2}

	\paragraph{Version finale}
