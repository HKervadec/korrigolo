\section{Organisation}
	\label{sec:livrables}

	%Au cours de nos réunions, nous nous sommes naturellement dirigés vers une méthode agile ressemblant à \og Scrum \fg. Nous nous réunissons une fois par semaine afin de discuter de l'avancement de nos tâches respectives et faire part de nos difficultés. Nous définissons en fonction de cela des tâches à effectuer la semaine suivante. Ainsi, nous avons décidé de livrer trois versions de \glasir{}, dont 2 intermédiaires.
	%Chaque version sera composée de sprints au cours desquels nous nous consacrerons à un sous-ensemble de tâches nécessaires à une version. %whattt ???
	%Nous présentons ci-dessous la répartition des livrables.

	
		% \paragraph{v 0.1 Paramètres de synthèse :}
	% 	\begin{description}
	% 		\item[sprint 1] %Implémenter l'IHM, y intégrer ADTool, créer un projet, l'afficher dans l'arborescence ;
	% 		\begin{description}
	% 	      \item[] 
	% 		  \item[] Implémenter l'IHM,
	% 		  \item[] intégrer ADTool,
	% 		  \item[] créer un projet, l'afficher dans l'arborescence ;
	% 		\end{description}
	% 		\item[sprint 2] 
	% 		\begin{description}
	% 	      \item[] 
	% 		  \item[] Évaluer une formule dans ADTool;
	% 		  \item[] la propager dans l'arbre;
	% 		  \item[] améliorer le codage des arbres.
	% 		\end{description}	
	% 		\item[sprint 3]
	% 		\begin{description}
	% 	      \item[] 
	% 		  \item[] Rendre visibles plusieurs paramètres;
	% 		  \item[] communiquer les formules entre ADTool et \glasir{};
	% 		  \item[]documenter le code et tester.
	% 		\end{description}	


	% 	\end{description}
	% \paragraph{v 0.2 filtre :} 
	% 	\begin{description}
	% 		\item[sprint 1] Définir l'algorithme de filtrage, créer l'IHM ;
	% 		\item[sprint 2] Créer un panneau dans \glasir{}, afficher l'arbre résultat, tester ;
	% 		\item[sprint 3] Documenter le code, ouverture simultanée de plusieurs arbres ;
	% 	\end{description}
	% \paragraph{v 1.0 optimiseur : }
	% 	\begin{description}
	% 		\item[sprint 1] Implémenter l'algorithme de calcul du chemin optimal, créer le panneau associé dans l'IHM ;
	% 		\item[sprint 2] Annuler une action, créer une bibliothèque de modèles, créer la page HTML ;
	% 		\item[sprint 3] Documenter le code, harmoniser les graphismes.
	% 	\end{description}


	\subsection{ Première version }

		La première version comprendra la création de la structure de l'interface graphique de Glasir et l'implémentation du module Editeur de Fonction. Elle sera réalisé en trois sprint. 


		\paragraph{sprint 1} Le premier sprint comprendra la création de l'IHM de base de Glasir, l'intégration du logiciel ADTool en tant que fenêtre de Glasir, la création d'un nouveau projet et sa sauvegarde, l'affichage des différents arbres d'un projet dans un dock sous forme d'arborescence.


		\paragraph{sprint 2} Le deuxième sprint contiendra la reconnaissance d'une formule mathématique par le logiciel ADTool, la création d'un paramètre en fonction de l'évaluation de cette formule, et l'amélioration de la représentation des arbres sous forme d'une grammaire au sein d'ADTool.


		\paragraph{sprint 3} Le troisième sprint comprendra la possibilité de représenter plusieurs paramètres par noeuds d'un arbre ainsi que la communication entre le module editeur de fonction de Glasir et la partie création d'un paramètre de synthése implémenter précédemment dans ADTool. 


	\subsection{ Deuxième version }

		La deuxième version permettra à l'utilisateur d'utiliser le module filtre. Elle sera, elle aussi, composé de trois sprint. 

		\paragraph{sprint 1}  le premier sprint consistera à implémenter l'algorithme de filtrage et à permettre l'ouverture de plusieurs instance d'ADTool au sein de différents onglet de l'IHM de Glasir. 

		\paragraph{sprint 2} Au sein du second sprint, le panneau contenant le module sera modélisé au sein de l'IHM de Glasir. De plus, la fonction couper/copier-coller sera implémentée dans ADTool. % ou durant le snd sprint?

		\paragraph{sprint 3} L'affichage de l'arbre filtré sera implémenté dans le troisième sprint. 


	\subsection{ Troisième version }

		La troisième version consistera en l'implémentation de la troisième fonctionnalité majeur de Glasir : l'optimiseur. 

		\paragraph{sprint 1} Le premier sprint comprendra l'implémentation de l'algorithme de l'optimiseur. Ainsi que la création d'une bibliothèque de modèles. 

		\paragraph{sprint 2} Le panneau contenant le module Optimiseur sera implémenter dans un second sprint. Dans un même temps, la possibilité d'annuler une action sera implémenter dans ADTool. 

		\paragraph{sprint 3} Le troisième sprint, aboutissant à la version final du logiciel, consistera en l'harmonisation de l'interface et la réalisation du packaging 


	Ce plan d'action est amené à changer au fur et à mesure de sa réalisation, ces évolutions étant le but même des méthodes dites \og agiles \fg.
	

