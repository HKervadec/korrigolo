\section{Répartition des tâches}
	\label{sec:repartition}

	Lors de la phase de développement de \glasir{}, notre groupe de projet sera réduit à trois étudiants : Pierre-Marie {\sc Airiau}, Valentin {\sc Esmieu} et Maud {\sc Leray}. Les trois autres, Florent {\sc Mallard}, Hoel {\sc Kervadec} ainsi que Corentin {\sc Nicole} partiront en effet étudier à l'étranger.
	
	Nous comptons donc nous répartir les tâches selon trois composantes générales : Maud L. travaillera principalement sur ADTool, Pierre-Marie A. sur les algorithmes et sur l'interfaçage entre ADTool et \glasir{}, et Valentin E. s'orientera sur l'interface utilisateur de \glasir{}. La {\sc Table} \ref{table:repartition} donne la répartition des tâches plus en détails.
		


	\begin{table}[H]
		\centering
		\begin{tabular}{|l|c|r||c|r||c|r|}
			\hline
			\multirow{2}{*}{} & \nomRepart{Pierre-Marie A.} & \nomRepart{Valentin E.} & \nomRepartt{Maud L.}\\
			\cline{2-7}
			 & {\bf Id tâche} & {\bf Durée} & {\bf Id tâche} & {\bf Durée} & {\bf Id tâche} & {\bf Durée}\\
			\hline
			{\bf Version 0.1} & - & {\bf 38h} & - & {\bf 34h} & - & {\bf 40h}\\
			 & 1.3 & 10h & 1.1 & 6h & 1.2 & 10h\\
			 & 1.4 & 12h & 1.2 & 10h & 1.6 & 18h\\
			 & 1.7 & 16h & 1.3 & 10h & 1.8 & 12h\\
			 & - & - & 1.5 & 8h & - & -\\
			\hline
			{\bf Version 0.2} & - & {\bf 39h} & - & {\bf 33h} & - & {\bf 28h}\\
			 & 2.1 & 24h & 2.3 & 20h & 2.4 & 16h\\
			 & 2.2 & 15h & 2.5 & 13h & 2.5 & 12h\\
			\hline
			{\bf Version 1.0} & - & {\bf 25h} & - & {\bf 23h} & - & {\bf 42h}\\
			 & 3.1 & 15h & 3.1 & 15h & 3.2 & 10h\\
			 & 3.2 & 10h & 3.4 & 8h & 3.3 & 16h\\
			 & - & - & - & - & 3.5 & 16h\\
			\hline
			{\bf Total} & \multicolumn{2}{r||}{{\bf 112h}} & \multicolumn{2}{r||}{{\bf 100h}} & \multicolumn{2}{r|}{{\bf 90h}}\\
			\hline
		\end{tabular}
		\caption{Répartition des tâches, par personne et par version.}
		\label{table:repartition}
		\label{tab:repartition}
	\end{table}
		