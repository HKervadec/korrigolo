\section{Répartition}
	\label{sec:repartition}

	Les tâches étant désormais bien définies, il est possible de les regrouper par domaine. Par exemple, toutes les améliorations apportées sur ADTool auront attrait à du code Java brut, tandis que ce qui concerne les fonctionnalités de calcul s'orienteront plus vers de l'algorithmique pure. Il est donc pertinent de définir ces catégories, puis de les répartir entre nous afin de se \og spécialiser \fg{} progressivement dans un domaine.
	
	Lors de la phase de développement de Glasir, notre groupe de projet sera réduit à trois étudiants : Pierre-Marie {\sc Airiau}, Valentin {\sc Esmieu} et Maud {\sc Leray}. Les trois autres, Florent {\sc Mallard}, Hoel {\sc Kervadec} ainsi que Corentin {\sc Nicole} partiront en effet étudier à l'étranger.
	
	Nous comptons donc nous répartir les tâches selon trois composantes générales : Maud travaillera sur ADTool, Pierre-Marie sur les algorithmes et sur l'interfaçage entre ADTool et Glasir, et Valentin s'orientera sur l'interface utilisateur de Glasir. Les {\sc Tables} \ref{table:repartition1}, \ref{table:repartition2} et \ref{table:repartition3} donnent la répartition des tâches en détails.
		
		\subsection{Version intermédiaire 1}
		\begin{table}[h]
			\centering
			\begin{tabular}{|c|r|c|}
				\hline
				\textbf{Étudiant} & \textbf{Tâches} & \textbf{Heures} \\
				\hline

			\end{tabular}
			\caption{Répartition des tâches version intermédiaire 1}
			\label{table:repartition1}
		\end{table}	
		
		\subsection{Version intermédiaire 2}
		\begin{table}[h]
			\centering
			\begin{tabular}{|c|r|l|c|r|}
				\hline
				\textbf{Étudiant} & \textbf{Tâches} & \textbf{Heures} \\
				\hline

			\end{tabular}
			\caption{Répartition des tâches version intermédiaire 1}
			\label{table:repartition2}
		\end{table}	
		
		\subsection{Version finale}
		\begin{table}[h]
			\centering
			\begin{tabular}{|c|r|l|c|r|}
				\hline
				\textbf{Étudiant} & \textbf{Tâches} & \textbf{Heures} \\
				\hline

			\end{tabular}
			\caption{Répartition des tâches version intermédiaire 1}
			\label{table:repartition3}
		\end{table}	