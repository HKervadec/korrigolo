\section{Risques}
	\label{sec:risques}

	L'implémentation d'un logiciel comporte toujours des risques. Afin de ne pas nous laisser surprendre par ces derniers, nous avons cherché à lister ceux qui nous venaient à l'esprit. Ils sont répertoriés dans le tableau\ref{fig:risques}, avec les tâches qu'ils concernent. Nous leur avons ajouté les notions de probabilité (Pr) et de criticité (Cr), sur une échelle de 1 à 3 : 1 pour faible, 2 pour moyenne, 3 pour élevée. Enfin, la dernière colonne cite des solutions possibles pour contre les risques énoncés.

	\begin{table}
		\centering
		\begin{tabular}{|p{4cm}|l|l|l|p{4cm}|}
		\hline
            \textbf{Risque} & \textbf{Tâches concernées} & \textbf{Pr.} & \textbf{Cr.} & \textbf{Solution}\\
            \hline
            Manque de temps pour tout réaliser & 
                Toutes & 3 & 3 &
                Bien s'organiser et respecter les délais\\ 
            \hline
            Apparition d'un bug difficile à corriger & 
                Toutes & 3 & 3 &
                Tests réguliers\\ 
            \hline
            Manque de connaissances techniques & 
                Toutes & 3 & 3 &
                Approfondir nos connaissances\\ 
            \hline
            Rédaction de la documentation à la fin & 
                Toutes & 2 & 2 &
                Rédaction au fur et à mesure du code\\
            \hline
            Mauvaise compréhension d'ADTool & 
                Toutes celles sur ADTool & 2 & 2 &
                Bien étudier le code d'ADTool et son fonctionnement\\ 
            \hline
            Perte de temps sur une tâche secondaire & 
                Toutes & 2 & 1 &
                Compter ses heures et faire des checkpoints\\ 
            \hline
            Algorithme ralentissant le logiciel & 
                3.1, 2.2 & 1 & 1 &
                Optimiser l’algorithme\\ 
            \hline
            Échec de l'intégration d'ADTool dans \glasir{} & 
                1.2 & 1 & 1 &
                Lancer ADTool séparément\\ 
            \hline
		\end{tabular}
		\caption{Tableau des risques}
		\label{fig:risques}
	\end{table}
