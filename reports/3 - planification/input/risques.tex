\section{Retour sur les risques}
\label{sec:risques}

Le rapport de planification~\cite{planif}, en plus de la répartition des tâches, comportait une section sur les risques prévisibles concernant le développement de Glasir. La présente section commence par rappeler ces risques dans la {\sc sous-section}~\ref{ssec:risquesBase} avant de comparer avec le déroulement réel de l'implémentation de Glasir dans la {\sc sous-section}~\ref{ssec:risquesRetro}.

\subsection{Rappel des risques pressentis}
\label{ssec:risquesBase}

Afin de délivrer ce projet dans les délais prévus, une évaluation des risques a été effectuée lors de la rédaction du rapport de planification~\cite{planif}. Pour chacun de ces risques, les notions de probabilité (Pr) et de criticité (Cr) ont été estimées. Les valeurs possibles sont les suivantes : L pour \og Low \fg{} (faible), M pour \og Medium \fg{} (moyenne), H pour \og High \fg{} (élevée). Dans le but de réagir au mieux en cas d’apparition de ces aléas, des solutions ont été définies. Le résultat de cette réflexion est présenté dans la \textsc{table}~\ref{fig:risques}. 

    \begin{table}[H]
        \centering
        \begin{tabular}{|p{4cm}|l|l|l|p{4cm}|}
        	\hline
            \textbf{Risque} & \textbf{Tâches concernées} & \textbf{Pr.} & \textbf{Cr.} & \textbf{Solution}\\
            \hline
            Mauvaise estimation des durées nécessaires aux tâches unitaires & 
                Toutes & H & H &
                Prioriser les tâches restantes\\ 
            \hline
            Apparition d'un bug difficile à corriger & 
                Toutes & H & H &
                Faire des tests unitaires\\
            \hline
            Difficulté à créer une grammaire & 
                1.4 & L & M &
                Simplifier les expressions à évaluer\\ 
            \hline
            Manque de connaissances techniques & 
                Toutes & H & H &
                Approfondir nos connaissances\\ 
            \hline
            Rédaction trop tardive de la documentation & 
                Toutes & M & L &
                Commenter le code au fur et à mesure\\
            \hline
            Mauvaise compréhension du code d'ADTool & 
                Toutes celles sur ADTool & M & M &
                Contacter le développeur d'ADTool\\ 
            \hline
            Mauvaise communication entre ADTool et Glasir & 
                1.4, 2.1, 3.1 & M & H &
                Utiliser le fichier XML lisible par ADTool\\ 
            \hline
            Perte de temps sur une tâche secondaire & 
                Toutes & M & L &
                Compter ses heures, faire le point lors des réunions\\ 
            \hline
            Algorithme ralentissant le logiciel & 
                3.1, 2.2 & L & L &
                Optimiser l’algorithme\\ 
            \hline
            Échec de l'intégration d'ADTool dans Glasir & 
                1.2 & L & H &
                Lancer ADTool séparément\\ 
            \hline
        \end{tabular}
        \caption{Tableau des risques}
        \label{fig:risques}
    \end{table}

Maintenant que l'implémentation de Glasir est terminée, il est intéressant de revenir sur les prévisions précédentes afin d'évaluer leur cohérence. 

\subsection{Rétrospective}
\label{ssec:risquesRetro}