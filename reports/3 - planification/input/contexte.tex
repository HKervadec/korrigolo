\section{Contexte}

	\subsection{Besoins et Objectifs}
	
	\subsection{Éléments en entrée}
	
	\subsection{Rétrospective et calendrier}
	%intégrer les dates à la rétrospective
	Nous avons déjà rédigé deux rapports dans le cadre de ce projet, nous permettant d'en aborder différents aspects. Leur contenu est brièvement rappelé ci-dessous.

	\paragraph{Rapport de pré-étude} Nous avons commencé ce projet par une étude de l'existant : la théorie des ADTrees ainsi que la découverte d'ADTool, un logiciel permettant d'en créer et d'en manipuler. Nous avons relevé lors de cette étude des lacunes empêchant une analyse poussée des ADTrees. Le contexte (transports en commun, plus particulièrement le STAR) a lui aussi été traité, afin de mieux cerner notre cas d'étude. Nous entendons par \og attaque \fg{} tout événement nuisant au bon fonctionnement du réseau de transport, il est donc apparu que les attaques contre les transports en commun sont assez fréquentes, même si elles ne sont pas toujours volontaires. Suite à cela, nous avons été en mesure de proposer un premier cahier des charges en date du 23 octobre. Ce dernier a été étoffé et précisé dans le second rapport.

	\paragraph{Rapport de spécifications fonctionnelles} Les fonctionnalités du logiciel ont été décrites de façon exhaustive. Cela nous a permis de fournir une première ébauche de l'architecture de \glasir{}, comprenant les dépendances entre les modules, et d'établir les différents liens les unissant. Cela nous a donné une vision plus globale et claire du projet, nous permettant de hiérarchiser les tâches à effectuer. Pour finir, nous avons proposé une planification succincte, développée dans le présent rapport. Celle-ci prévoyait la séparation du développement en de nombreuses versions, et a été remise le 27 novembre.

	Ces rapports nous ont aidé à préciser notre vision du projet, à identifier les modules de la solution que nous apporterons et définir les étapes dans sa réalisation.

	Dans le cahier des charges commun à tous les projets, il figure des dates-clés indiquant un rapport à rendre ou une présentation orale. Nous allons donc rappeler ici les échéances que nous devrons respecter. La phase de conception logicielle commencera dès les vacances de Noël le 20 décembre. Un rapport terminera cette phase le 12 février. Une page HTML sera livrée le 2 avril, et le projet final sera livré sur CD-Rom le 28 mai et nous ferons une démonstration de son fonctionnement.
	Nous pourrons profiter de semaines \og bloquées\fg{}, pendant lesquelles nous n'aurons aucun cours, pour nous concentrer pleinement à \glasir{}. Les semaines du 18 au 23 mai et du 28 au 28 mai sont donc exemptes de cours. 

Nous allons maintenant décrire la méthode de travail que nous avons décidé d'utiliser, ainsi que l'organisation que nous avons adoptée au sein du groupe.