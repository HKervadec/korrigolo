\section{Tâches unitaires}
	\label{sec:taches_unitaires}

	Nous avons décidé de livrer trois versions de \glasir{} : deux intermédiaires et une finale. Ces dernières ont elles-mêmes été divisées en tâches unitaires, décrites dans cette section. 
	
	Il est à noter que les tests et la rédaction de la documentation sont pris en compte dans l'estimation du temps nécessaire à la réalisation de chacune des tâches.
	

	\subsection{Version 0.1}
		Huit grandes tâches ont été identifiées pour la version 0.1 de notre logiciel. Celles-ci sont présentées dans la {\sc Table} \ref{tab:taches_units_1}. 
		\begin{table}[H]
			\centering
			\begin{tabular}{|c|r|l|c|r|}
				\hline
				\textbf{Cible} & \textbf{Id} & \textbf{Tâche} & \textbf{Technologies} & \textbf{Durée}\\
				\hline

				\multirow{5}{*}{\glasir{}} & 1.1 & Création squelette interface & WPF & 6h\\
				\cline{2-5}
				 & 1.2 & Gestion fichiers projet & C++ & 20h\\
				\cline{2-5}
				 & 1.3 & Intégration ADTool dans \glasir & JNI & 20h\\
				\cline{2-5}
				 & 1.4 & \'Evaluateur de fonction & Java & 12h\\
				\cline{2-5}
				 & 1.5 & Interface évaluateur & WPF & 8h\\
				\hline

				\multirow{3}{*}{ADTool} & 1.6 & Valuation ADTrees & \multirow{3}{*}{Java} & 18h\\
				\cline{2-3} \cline{5-5}
				 & 1.7 & Refonte langage des ADTrees & & 16h\\
				\cline{2-3} \cline{5-5}
				 & 1.8 & Vue globale des paramètres & & 12h\\
				\hline

				\multicolumn{4}{|l|}{\bf Total} & {\bf 112h}\\
				\hline
			\end{tabular}
			\caption{Tâches associées à la version 0.1.}
			\label{tab:taches_units_1}
		\end{table}
		
		
		La réalisation de \glasir{} 0.1 sera découpée en trois sprints, présentés ci-dessous.\newline
		\textbf{Sprint 1} Création de l'interface utilisateur de base de \glasir{}, intégration du logiciel ADTool en tant que fenêtre de \glasir{}, possibilité de création d'un nouveau projet et sa sauvegarde, affichage des différents arbres d'un projet dans un dock sous forme d'arborescence.\newline
		\textbf{Sprint 2} Reconnaissance d'une formule mathématique par le logiciel ADTool, création d'un paramètre en fonction de l'évaluation de cette formule, et l'amélioration de la représentation des arbres sous forme d'une grammaire au sein d'ADTool.\newline
		\textbf{Sprint 3} Affichage de plusieurs paramètres par nœud d'un arbre, possibilité de communication entre le module éditeur de fonction de \glasir{} et la partie création d'un paramètre de synthèse implémentée précédemment dans ADTool.\newline


	\subsection{Version 0.2}
		Le développement de la version 0.2 est découpé en cinq tâches, visibles dans la {\sc table} \ref{tab:taches_units_2}.
		\begin{table}[h]
			\centering
			\begin{tabular}{|c|r|l|c|r|}
				\hline
				\textbf{Cible} & \textbf{Id} & \textbf{Tâche} & \textbf{Technologies} & \textbf{Durée}\\
				\hline

				\multirow{4}{*}{\glasir{}} & 2.1 & Algorithme filtrage & C++ & 24h\\
				\cline{2-5}
				 & 2.2 & Interface filtre & WPF & 15h\\
				\cline{2-5}
				 & 2.3 & Multiples instances d'ADTool & C++, WPF & 20h\\
				\cline{2-5}
				 & 2.4 & Affichage arbre filtré & Java, WPF & 16h\\
				\hline

				\multirow{1}{*}{ADTool} & 2.5 & Couper/copier/coller & \multirow{1}{*}{Java} & 25h\\
				\hline

				\multicolumn{4}{|l|}{\bf Total} & {\bf 100h}\\
				\hline
			\end{tabular}
			\caption{Tâches associées à la version 0.2.}
			\label{tab:taches_units_2}
		\end{table}
		
		Les sprints se composeront de :
		
		\textbf{Sprint 4} Implémentation de l'algorithme de filtrage, permettre l'ouverture de plusieurs instances d'ADTool au sein de différents onglets de l'interface de \glasir{}.\newline 
		\textbf{Sprint 5} Affichage du panneau relatif à ce module au sein de l'interface utilisateur de \glasir{}. Implémentation de la fonction couper/copier-coller.\newline % ou durant le snd sprint?
		\textbf{Sprint 6} Affichage de l'arbre filtré.

	\subsection{Version 1.0}
		Nous avons identifié quatre tâches pour la version 1.0 de \glasir{}, regroupées dans la {\sc Table} \ref{tab:taches_units_3}.
		\begin{table}[h]
			\centering
			\begin{tabular}{|c|r|l|c|r|}
				\hline
				\textbf{Cible} & \textbf{Id} & \textbf{Tâche} & \textbf{Technologies} & \textbf{Durée}\\
				\hline

				\multirow{4}{*}{\glasir{}} & 3.1 & Optimiseur & C++, WPF & 30h\\
				\cline{2-5}
				 & 3.2 & Bibliothèque de modèles & C++, WPF & 20h\\
				\cline{2-5}
				 & 3.3 & Harmonisation interface & WPF & 16h\\
				\cline{2-5}
				 & 3.4 & Packaging & ? & 8h\\
				\hline

				\multirow{1}{*}{ADTool} & 3.5 & Ctrl-Z & \multirow{1}{*}{Java} & 16h\\
				\hline

				\multicolumn{4}{|l|}{\bf Total} & {\bf 90h}\\
				\hline
			\end{tabular}
			\caption{Tâches associées à la version 1.0.}
			\label{tab:taches_units_3}
		\end{table}
		
		Les sprints se composeront de :		
		
		\textbf{Sprint 7} Implémentation de l'algorithme de l'optimiseur, création d'une bibliothèque de modèles.\newline
		\textbf{Sprint 8} Affichage du panneau de l'optimiseur dans \glasir{}, possibilité d'annuler une action dans ADTool.\newline
		\textbf{Sprint 9} Harmonisation de l'interface et réalisation du packaging. \newline