\section{Organisation}
    \label{sec:orga}
    Cette section va à présent décrire la méthode de travail que nous avons décidé d'utiliser, ainsi que l'organisation que nous avons adoptée au sein du groupe.

    \subsection{La méthode SCRUM}
    \label{subsec:scrum}
        Afin de définir la méthode de gestion de projet à mettre en place pour la réalisation de \glasir{}, nous avons débattu ensemble des différentes méthodes à notre disposition. A la suite de     nos discussions en réunions, nous nous sommes dirigés vers une méthode agile proche du \og \textsc{scrum} \fg{}. Ce choix s'est fait sur les points qui sont détaillé ci-dessous :

        L'organisation des rôles que nous avons mise en place depuis le début du projet se base sur un rôle de \og Coordinateur \fg{}, qui est responsable de la réalisation et du rendu d'un livrable (rapport, version logicielle, etc). Ce rôle tourne à chaque nouveau livrable, de sorte que chacun des membres de l'équipe l'exerce au moins une fois. Le rôle de coordinateur que nous avons défini nous a donc semblé correspondre au rôle de ScrumMaster tel que défini dans la méthode \textsc{scrum}.

        Le coordinateur endosse également la responsabilité de Product Owner, qu'il partage avec les encadrants du Projet pour ce qui est de la définition des Objectifs et surtout de l'acceptation des livrables.

        En place de la mêlée quotidienne du \textsc{scrum}, nous nous réunissons une fois par semaine afin de discuter de l'avancement de nos tâches respectives et faire part de nos difficultés. Nous avons décidé d'adopter un rythme hebdomadaire plutôt que quotidien car nous ne travaillerons pas à temps plein sur le projet et ce rythme nous a donc semblé plus adéquat. Nous définissons en fonction des retours en réunion les éventuels ajustements à effectuer la semaine suivante et dans le reste de la planification.

        Enfin, chaque version sera composée de sprints au cours desquels nous nous consacrerons à un sous-ensemble de tâches nécessaires à la réalisation d'une version de \glasir{}. Ces tâches constitueront les Users Stories du projet, et sont détaillées dans la \textsc{sous-section}~\ref{subsec:taches_unitaires}.  
    
    \subsection{Tâches unitaires et estimations}
    \label{subsec:taches_unitaires}

        Trois versions de \glasir{} seront livrées : deux intermédiaires et une finale. Le développement de chacune de ces versions a été divisé en tâches unitaires, détaillées dans les {\sc Tables} \ref{tab:taches_units_1}, \ref{tab:taches_units_2} et \ref{tab:taches_units_3}.
        
        Pour déterminer le temps nécessaire à l'accomplissement de chaque tâche, nous avons combiné les méthodes d’estimation \og Analogique \fg{} et \og Expertise \fg{}. En effet, la méthode \og Analogique \fg{} consiste à estimer la durée d'une tâche par analogie avec une tâche similaire déjà effectuée (au sein d'un autre projet par exemple). La méthode \og Expertise \fg{}, quant à elle, repose sur le jugement d'un expert. Bien qu'aucun membre de notre groupe ne puisse réellement se considérer comme tel, nos stages respectifs ont tout de même permis d'apporter des avis constructifs sur l'estimation de la durée des tâches.
        
        Les résultats de ces différentes estimations sont décrits ci-après pour chaque version. Il est à noter que les tests et la rédaction de la documentation sont pris en compte dans l'estimation du temps nécessaire à la réalisation de chacune des tâches.
        

        \subsubsection{Version 0.1}
            Huit grandes tâches ont été identifiées pour cette version et sont présentées dans la {\sc Table} \ref{tab:taches_units_1}. 
            \begin{table}[H]
                \centering
                \begin{tabular}{|c|r|l|c|r|}
                    \hline
                    \textbf{Cible} & \textbf{Id} & \textbf{Tâche} & \textbf{Technologies} & \textbf{Durée}\\
                    \hline

                    \multirow{5}{*}{\glasir{}} & 1.1 & Squelette interface & WPF & 6h\\
                    \cline{2-5}
                     & 1.2 & Gestion fichiers projet & C++ & 20h\\
                    \cline{2-5}
                     & 1.3 & Intégration ADTool dans \glasir & JNI & 20h\\
                    \cline{2-5}
                     & 1.4 & \'Evaluateur de fonction & Java & 12h\\
                    \cline{2-5}
                     & 1.5 & Interface évaluateur & WPF & 8h\\
                    \hline

                    \multirow{3}{*}{ADTool} & 1.6 & Valuation ADTrees & \multirow{3}{*}{Java} & 18h\\
                    \cline{2-3} \cline{5-5}
                     & 1.7 & Refonte langage des ADTrees & & 16h\\
                    \cline{2-3} \cline{5-5}
                     & 1.8 & Vue globale des paramètres & & 12h\\
                    \hline

                    \multicolumn{4}{|l|}{\bf Total} & {\bf 112h}\\
                    \hline
                \end{tabular}
                \caption{Tâches associées au développement de \glasir{} version 0.1.}
                \label{tab:taches_units_1}
            \end{table}
            
            La réalisation de cette version sera découpée en trois sprints, présentés ci-dessous :

            \noindent\textbf{Sprint 1} Création de l'interface utilisateur de base de \glasir{}, intégration du logiciel ADTool en tant que fenêtre de \glasir{}, possibilité de création d'un nouveau projet et sa sauvegarde, affichage des différents arbres d'un projet dans un dock sous forme d'arborescence.\newline
            \textbf{Sprint 2} Reconnaissance d'une formule mathématique par le logiciel ADTool, création d'un paramètre en fonction de l'évaluation de cette formule, et l'amélioration de la représentation des arbres sous forme d'une grammaire au sein d'ADTool.\newline
            \textbf{Sprint 3} Affichage de plusieurs paramètres par nœud d'un arbre, possibilité de communication entre le module éditeur de fonction de \glasir{} et la partie création d'un paramètre de synthèse implémentée précédemment dans ADTool.\newline


        \subsubsection{Version 0.2}
            Le développement de la version 0.2 est découpé en cinq tâches, résumées dans la {\sc table} \ref{tab:taches_units_2}.
            \begin{table}[h]
                \centering
                \begin{tabular}{|c|r|l|c|r|}
                    \hline
                    \textbf{Cible} & \textbf{Id} & \textbf{Tâche} & \textbf{Technologies} & \textbf{Durée}\\
                    \hline

                    \multirow{4}{*}{\glasir{}} & 2.1 & Algorithme filtrage & C++ & 24h\\
                    \cline{2-5}
                     & 2.2 & Interface filtre & WPF & 15h\\
                    \cline{2-5}
                     & 2.3 & Multiples instances d'ADTool & C++, WPF & 20h\\
                    \cline{2-5}
                     & 2.4 & Affichage arbre filtré & Java, WPF & 16h\\
                    \hline

                    \multirow{1}{*}{ADTool} & 2.5 & Couper/copier/coller & \multirow{1}{*}{Java} & 25h\\
                    \hline

                    \multicolumn{4}{|l|}{\bf Total} & {\bf 100h}\\
                    \hline
                \end{tabular}
                \caption{Tâches associées au développement de \glasir{} version 0.2.}
                \label{tab:taches_units_2}
            \end{table}
            
            Nous avons réparti les tâches en trois sprints :

            \noindent\textbf{Sprint 4} Implémentation de l'algorithme de filtrage, possibilité d'ouverture de plusieurs instances d'ADTool au sein de différents onglets de l'interface de \glasir{}.\newline 
            \textbf{Sprint 5} Affichage du panneau relatif à ce module au sein de l'interface utilisateur de \glasir{}. Implémentation de la fonction couper/copier/coller.\newline % ou durant le snd sprint?
            \textbf{Sprint 6} Affichage de l'arbre filtré.

        \subsubsection{Version 1.0}
            Quatre tâches ont été identifiées pour la version 1.0 de \glasir{}, présentées dans la {\sc Table} \ref{tab:taches_units_3}.
            \begin{table}[h]
                \centering
                \begin{tabular}{|c|r|l|c|r|}
                    \hline
                    \textbf{Cible} & \textbf{Id} & \textbf{Tâche} & \textbf{Technologies} & \textbf{Durée}\\
                    \hline

                    \multirow{4}{*}{\glasir{}} & 3.1 & Optimiseur & C++, WPF & 30h\\
                    \cline{2-5}
                     & 3.2 & Bibliothèque de modèles & C++, WPF & 20h\\
                    \cline{2-5}
                     & 3.3 & Harmonisation interface & WPF & 16h\\
                    \cline{2-5}
                     & 3.4 & Packaging & ? & 8h\\
                    \hline

                    \multirow{1}{*}{ADTool} & 3.5 & Ctrl-Z & \multirow{1}{*}{Java} & 16h\\
                    \hline

                    \multicolumn{4}{|l|}{\bf Total} & {\bf 90h}\\
                    \hline
                \end{tabular}
                \caption{Tâches associées au développement de \glasir{} version 1.0.}
                \label{tab:taches_units_3}
            \end{table}
            
            Les sprints se décomposeront de la façon suivante :

            \noindent\textbf{Sprint 7} Implémentation de l'algorithme de l'optimiseur, création d'une bibliothèque de modèles.\newline
            \textbf{Sprint 8} Affichage du panneau de l'optimiseur dans \glasir{}, possibilité d'annuler une action dans ADTool.\newline
            \textbf{Sprint 9} Harmonisation de l'interface, et réalisation du packaging. \newline

        %TODO estimations


    \subsection{Risques}
    \label{subsec:risques}  
        Dans le but de délivrer ce projet dans les délais, une évaluation des risques a été effectuée. Celle-ci a pour but d'identifier les différents facteurs qui pourraient engendrer des difficultés dans la réalisation des tâches et entraîner ainsi un retard, ou une incapacité à implémenter une partie du cahier des charges. Nous avons associé à chaque élément de cette liste les notions de probabilité (Pr) et de criticité (Cr), sur une échelle de 1 à 3 : 1 pour faible, 2 pour moyenne, 3 pour élevée. Dans le but de réagir au mieux en cas d’apparition de ces aléas, des solutions ont été définies. Le résultat de cette réflexion est présenté dans la \textsc{table}~\ref{fig:risques}. 


    \begin{table}[H]
        \centering
        \begin{tabular}{|p{4cm}|l|l|l|p{4cm}|}
        	\hline
            \textbf{Risque} & \textbf{Tâches concernées} & \textbf{Pr.} & \textbf{Cr.} & \textbf{Solution}\\
            \hline
            Mauvaise estimation des durées nécessaires aux tâches unitaires & 
                Toutes & 3 & 3 &
                Prioriser les tâches restantes\\ 
            \hline
            Apparition d'un bug difficile à corriger & 
                Toutes & 3 & 3 &
                Faire des tests unitaires\\
            \hline
            Difficulté à créer une grammaire & 
                1.4 & 1 & 2 &
                Simplifier les expressions à évaluer\\ 
            \hline
            Manque de connaissances techniques & 
                Toutes & 3 & 3 &
                Approfondir nos connaissances\\ 
            \hline
            Rédaction trop tardive de la documentation & 
                Toutes & 2 & 1 &
                Commenter le code au fur et à mesure\\
            \hline
            Mauvaise compréhension du code d'ADTool & 
                Toutes celles sur ADTool & 2 & 2 &
                Contacter le développeur d'ADTool\\ 
            \hline
            Mauvaise communication entre ADTool et \glasir{} & 
                1.4, 2.1, 3.1 & 2 & 3 &
                Utiliser le fichier XML lisible par ADTool\\ 
            \hline
            Perte de temps sur une tâche secondaire & 
                Toutes & 2 & 1 &
                Compter ses heures, faire le point lors des réunions\\ 
            \hline
            Algorithme ralentissant le logiciel & 
                3.1, 2.2 & 1 & 1 &
                Optimiser l’algorithme\\ 
            \hline
            Échec de l'intégration d'ADTool dans \glasir{} & 
                1.2 & 1 & 3 &
                Lancer ADTool séparément\\ 
            \hline
        \end{tabular}
        \caption{Tableau des risques}
        \label{fig:risques}
    \end{table}
    
    \subsection{Organisation du groupe et répartition des tâches}
        Lors de la phase de développement de \glasir{}, notre groupe de projet sera réduit à trois étudiants : Pierre-Marie {\sc Airiau}, Valentin {\sc Esmieu} et Maud {\sc Leray}. Les trois autres, Florent {\sc Mallard}, Hoel {\sc Kervadec} ainsi que Corentin {\sc Nicole} partiront en effet étudier à l'étranger.
        
        Nous comptons donc nous répartir les tâches selon trois composantes générales : Maud L. travaillera principalement sur ADTool, Pierre-Marie A. sur les algorithmes et sur l'interfaçage entre ADTool et \glasir{}, et Valentin E. s'orientera sur l'interface utilisateur de \glasir{}. La {\sc Table} \ref{table:repartition} donne la répartition des tâches plus en détails.

        \begin{table}[H]
            \centering
            \begin{tabular}{|l|c|r||c|r||c|r|}
                \hline
                \multirow{2}{*}{} & \nomRepart{Pierre-Marie A.} & \nomRepart{Valentin E.} & \nomRepartt{Maud L.}\\
                \cline{2-7}
                 & {\bf Id tâche} & {\bf Durée} & {\bf Id tâche} & {\bf Durée} & {\bf Id tâche} & {\bf Durée}\\
                \hline
                {\bf Version 0.1} & - & {\bf 38h} & - & {\bf 34h} & - & {\bf 40h}\\
                 & 1.3 & 10h & 1.1 & 6h & 1.2 & 10h\\
                 & 1.4 & 12h & 1.2 & 10h & 1.6 & 18h\\
                 & 1.7 & 16h & 1.3 & 10h & 1.8 & 12h\\
                 & - & - & 1.5 & 8h & - & -\\
                \hline
                {\bf Version 0.2} & - & {\bf 39h} & - & {\bf 33h} & - & {\bf 28h}\\
                 & 2.1 & 24h & 2.3 & 20h & 2.4 & 16h\\
                 & 2.2 & 15h & 2.5 & 13h & 2.5 & 12h\\
                \hline
                {\bf Version 1.0} & - & {\bf 25h} & - & {\bf 23h} & - & {\bf 42h}\\
                 & 3.1 & 15h & 3.1 & 15h & 3.2 & 10h\\
                 & 3.2 & 10h & 3.4 & 8h & 3.3 & 16h\\
                 & - & - & - & - & 3.5 & 16h\\
                \hline
                {\bf Total} & \multicolumn{2}{r||}{{\bf 102h}} & \multicolumn{2}{r||}{{\bf 100h}} & \multicolumn{2}{r|}{{\bf 110h}}\\
                \hline
            \end{tabular}
            \caption{Répartition des tâches, par personne et par version.}
            \label{table:repartition}
            \label{tab:repartition}
        \end{table}
