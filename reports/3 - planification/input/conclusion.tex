\section{Conclusion}
    \label{sec:conclusion}

    Nous avons commencé par lister les tâches que nous allons devoir réaliser pour que notre projet soit mené à son terme. Nous les avons ensuite regroupées selon la version pour laquelle elles seront effectuées. Le développement de chacune des trois versions définies a quant à lui été divisé en sprints. Ces derniers sont caractéristiques de la méthode Scrum, méthode agile à laquelle nous avons aussi emprunté le concept de réunion de concertation sur l'avancement, même si dans notre cas elles ne sont qu'hebdomadaires. Nous avons ensuite envisagé les différents obstacles qui pourraient nous retarder, ce qui nous a permis d'estimer une durée nécessaire à chaque tâche en tenant compte de ces risques. Notre planification est le résultat de ces estimations. Elle sera soumise à des changements, mais ceci fait partie intégrante des méthodes agiles. Nous pouvons maintenant commencer la conception de \glasir{}, dans laquelle seront détaillées son architecture et son interface graphique.