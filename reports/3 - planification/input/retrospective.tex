\section{Rétrospective}
	\label{sec:retrospective}

	Nous avons déjà rédigé deux rapports dans le cadre de ce projet, nous permettant d'en aborder différents aspects. Leur contenu est brièvement rappelé ci-dessous.

	\paragraph{Rapport de pré-étude} Nous avons commencé ce projet par une étude de l'existant : la théorie des ADTrees ainsi que la découverte d'ADTool, un logiciel permettant d'en créer et d'en manipuler. Nous avons relevé lors de cette étude des lacunes empêchant une analyse poussée des ADTrees. Le contexte (transports en commun, plus particulièrement le STAR) a lui aussi été traité, afin de mieux cerner le problème. Il est apparu que les attaques contre les transports en commun sont assez fréquentes, qu'elles soient volontaires ou non. Nous entendons par \og attaque \fg tout événement nuisant au bon fonctionnement du réseau de transport. Suite à cela, nous avons été en mesure de proposer un premier cahier des charges. Ce dernier a été étoffé et précisé dans le second rapport.

	\paragraph{Rapport de spécifications fonctionnelles} Les fonctionnalités du logiciel ont été décrites de façon exhaustive. Cela nous a permis de fournir une première ébauche de l'architecture de \glasir{}, comprenant les dépendances entre les modules, et d'établir les différents liens les unissant. Cela nous a donné une vision plus globale et claire du projet, nous permettant de hiérarchiser les tâches à effectuer. Pour finir, nous avons proposé une planification succincte, développée dans le présent rapport. Celle-ci prévoyait la séparation du développement en de nombreuses versions.

	Celles-ci ont évolué depuis, et sont présentées en détail dans la section suivante.