\section{Séquencement}
	\label{sec:sequencement}
	Pour établir notre planification, nous avons utilisé le logiciel MS Project. Les diagrammes présentés dans cette section tiennent compte des différents jalons qui ont été posés, soit imposés à tous les groupes de projet de 4INFO, soit définis par nous-même.

	\subsection{Diagramme de Gantt}
		Le diagramme de Gantt visible sur la \ffigure{} \ref{fig:gantt} illustre le séquencement de notre projet. Les tâches ont été attribuées à une personne, et les durées ont été calculées pour une moyenne d'une heure et demi de travail personnel par jour. Les semaines du 18 et 25 mai, bloquées pour les 4INFO, la durée du travail personnel a bien sûr été augmentée. Si une tâche est réalisée par plusieurs personnes, les contributions de ces différentes personnes sont séparées par des pointillés.  
	
	\subsection{Répartition de la charge de travail}
		Le planning de la \ffigure{} \ref{fig:planning_charge} illustre la répartition individuelle du temps de réalisation des tâches ainsi que le cumul des heures de travail pour chacun. Nous avons essayé d'être aussi homogènes que possible dans la répartition du temps de travail, les différences de répartition étant la conséquence de l'irrégularité des durées des tâches.

	\subsection{Utilisation des ressources}
		Le dernier graphique à la \ffigure{} \ref{fig:taux_utilisation} rend compte de l'utilisation du temps de travail pour l'ensemble du groupe, que nous avons fixé à raison d'une heure et demi par jour en semaine normale. On peut distinguer les trois versions grâce au taux d'occupation des ressources. En effet il est moindre une semaine entre chaque version, car nous prévoyons du temps pour rattraper d'éventuels retards et effectuer quelques corrections.

		\begin{landscape}
		 	\begin{figure}
	            \centering
	            \includegraphics[height=0.70\textwidth]{figure/DiagGantt.png}
	            \caption{Diagramme de Gantt présentant la chronologie des tâches.}
	            \label{fig:gantt}
	        \end{figure}
	    \end{landscape}

		\begin{landscape}
		 	\begin{figure}
	            \centering
	            \includegraphics[height=0.70\textwidth]{figure/RepartitionTaches2.png}
	            \caption{Planning des charges réparties par personne}
	            \label{fig:planning_charge}
	        \end{figure}
	    \end{landscape}

		\begin{landscape}
		 	\begin{figure}
	            \centering
	            \includegraphics[height=0.70\textwidth]{figure/TauxUtilisation.png}
	            \caption{Taux d'utilisation global des ressources}
	            \label{fig:taux_utilisation}
	        \end{figure}
	    \end{landscape}