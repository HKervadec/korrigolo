\section{Introduction}
	\label{sec:intro}

	% ...

	% Glasir est l'application que nous allons réaliser. Basé sur l'éditeur d'arbres d'attaque et de défense (ADTrees) ADTool, il permettra de faciliter l'analyse d'ADTrees pour un expert en sécurité. En plus de réelles innovations apportant des fonctionnalités révolutionnaires, nous intégrerons des améliorations mineures à l'existant, afin de simplifier la vie d'un expert. L'utilité de notre logiciel sera illustrée par l'analyse du Service des Transports en commun de l'Agglomération Rennaise (STAR).

	% ...

	% Ce rapport contiendra...................

	Pour ce projet, nous devons créer un logiciel complet d'analyse de sécurité par les arbres d'attaque et de défense (ADTrees). Afin d'illustrer son fonctionnement, nous nous baserons sur un cas réel : le Service des Transports de l'Agglomération Rennaise (STAR).

	Un logiciel existe déjà pour ce type d'analyse (ADTool), et nous avons commencé par étudier son fonctionnement. Il a ensuite été décidé de l'intégrer à \glasir{} en apportant de nouvelles fonctionnalités et en améliorant celles qu'ADTool offre déjà. Cette réflexion sur le cahier des charges a constitué la première phase de notre projet. Lors de la seconde, nous avons défini plus précisément les différentes fonctionnalités que nous avions annoncées, et étudié les moyens de les implémenter. De cette manière, nous avons pu décrire comment elles allaient interagir.

	Ce rapport entame une autre phase d'analyse qui comprend la définition et la planification des différentes tâches à réaliser. Une estimation de leur durée et des risques associés y figureront également.