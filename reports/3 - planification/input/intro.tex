\section{Introduction}
	\label{sec:intro}

	% ...

	% Glasir est l'application que nous allons réaliser. Basé sur l'éditeur d'arbres d'attaque et de défense (ADTrees) ADTool, il permettra de faciliter l'analyse d'ADTrees pour un expert en sécurité. En plus de réelles innovations apportant des fonctionnalités révolutionnaires, nous intégrerons des améliorations mineures à l'existant, afin de simplifier la vie d'un expert. L'utilité de notre logiciel sera illustrée par l'analyse du Service des Transports en commun de l'Agglomération Rennaise (STAR).

	% ...

	% Ce rapport contiendra...................

	Le projet est de créer un logiciel complet d'analyse de sécurité par les arbres d'attaque et de défense. Afin d'illustrer son fonctionnement, nous avons choisi de l'illustrer sur un cas aussi réel que possible, le Service des Transports de l'Agglomération Rennaise(STAR).

	Un logiciel existant déjà pour ce type d'analyse, nous avons commencé par étudier son fonctionnement. Nous avons ensuite décidé de l'intégrer au nôtre, \glasir{}, afin de lui apporter de nouvelles fonctionnalités et améliorer celles déjà en place. Cette réflexion sur le cahier des charges a constitué la première phase de notre projet. Lors de la seconde, nous avons défini plus précisément les différentes fonctionnalités que nous avions annoncé, étudié les moyens de les implémenter, et avons de cette manière pu décrire comment elles interagiraient.

	Nous pouvons maintenant commencer une autre phase d'analyse, l'organisation et la planification des différentes tâches à réaliser. Ce rapport les présentera, accompagnées du la durée que nous estimons nécessaire à leur réalisation. Nous utiliserons pour cela le logiciel Microsoft MS Project. Un diagramme de Gantt illustrera l'ordonnancement, tandis que nous rassemblerons les risques liés à notre projet et leur impact sur celui-ci sous la forme d'un tableau.