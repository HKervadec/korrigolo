\section{Introduction}
	\label{sec:intro}

	% ...

	% Glasir est l'application que nous allons réaliser. Basé sur l'éditeur d'arbres d'attaque et de défense (ADTrees) ADTool, il permettra de faciliter l'analyse d'ADTrees pour un expert en sécurité. En plus de réelles innovations apportant des fonctionnalités révolutionnaires, nous intégrerons des améliorations mineures à l'existant, afin de simplifier la vie d'un expert. L'utilité de notre logiciel sera illustrée par l'analyse du Service des Transports en commun de l'Agglomération Rennaise (STAR).

	% ...

	% Ce rapport contiendra...................

	% Pour ce projet, nous devons créer un logiciel complet d'analyse de sécurité par les arbres d'attaque et de défense (ADTrees). Afin d'illustrer son fonctionnement, nous nous baserons sur un cas réel : le Service des Transports de l'Agglomération Rennaise (STAR).

	% Un logiciel existe déjà pour ce type d'analyse (ADTool), et nous avons commencé par étudier son fonctionnement. Il a ensuite été décidé de l'intégrer à \glasir{} en apportant de nouvelles fonctionnalités et en améliorant celles qu'ADTool offre déjà. Cette réflexion sur le cahier des charges a constitué la première phase de notre projet. Lors de la seconde, nous avons défini plus précisément les différentes fonctionnalités que nous avions annoncées, et étudié les moyens de les implémenter. De cette manière, nous avons pu décrire comment elles allaient interagir.

	% Ce rapport entame une autre phase d'analyse qui comprend la définition et la planification des différentes tâches à réaliser. Une estimation de leur durée et des risques associés y figureront également.

La sécurisation des systèmes est une problématique majeure de la société moderne. En ce sens, de nombreuses méthodologies ont été développées dans le but d'identifier les risques et de les quantifier. C'est avec cet objectif que le concept d'arbres d'attaque et de défense (\og Attack-Defense Trees \fg{} en anglais, ou ADTrees) a vu le jour.

Lors de la phase de pré-étude, nous avons pu comprendre l’intérêt pratique de la construction des ADTrees. Leur utilisation permet d'identifier de manière précise les différentes attaques possibles contre un système et de les valuer en termes de coût, de probabilité, etc. ADTool (Attack-Defense Tree Tool), un logiciel développé pour l'implémentation de ces arbres sur support informatique, a été étudié pour ce projet. Lors de sa prise en main, nous avons trouvé que le logiciel était limité. En effet, dans un cas concret d'expertise en sécurité, l'ADTree qui modélisera les attaques sur le système sera de très grande taille car leur nombre peut être très élevé. Dans ce cas, il est difficile pour l'expert d'en extraire des informations pertinentes au premier coup d’œil. Or, ADTool ne fournit pas d'outil permettant à l'utilisateur de simplifier l'analyse de l'arbre. Nous avons donc décidé de l'intégrer dans un autre logiciel.

Durant la phase de spécification, nous avons pu préciser le cahier des charges de notre projet. Celui-ci aboutira à la création d'un logiciel nommé \glasir{}. ADTool sera intégré dans une fenêtre au sein du logiciel et sera utilisé comme éditeur d'arbre. \glasir{} permettra à l'utilisateur de gérer son projet de modélisation des risques d'attaques à l'aide d'ADTrees et proposera trois fonctionnalités principales d'analyse des ADTrees : le paramètre de synthèse, le filtre et l'optimiseur.

Ce rapport entame une autre phase d'analyse. Dans un premier temps, il reprécise le contexte de ce projet. Dans un second temps, il détaille la méthode de gestion de projet utilisée et son partitionnement en tâches unitaires. Enfin, il précise, grâce à des diagrammes réalisés avec le logiciel MS Project, le séquencement de ces tâches durant le second trimestre.