\chapter{Cahier des charges}
    L'objectif principal de ce projet est de réaliser une suite logicielle, permettant à un utilisateur novice dans le domaine des arbres d'attaque défense et de la sécurité en général, d'évaluer les risques d'attaque sur son système.
    
    Notre suite contiendra:
    \begin{itemize}
    	\item Une bibliothèque de modèles d'attaques existantes.
        \item Base de valeurs.
        \item D'un guide (plus ou moins interactif) pour partir de zéro, même novice.
        \item Un editeur d'arbres.
    \end{itemize}
    
    \section{Bibliothèque d'attaques}    
        La bibliothèque d'attaques servirait à décrire des cas très généraux, que l'utilisateur pourrait détailler en fonction de sa situation. 
        
        Dans notre exemple (la STAR), nous pourrons fournir des arbres d'attaque de réseaux de transport communs à toutes les villes de France, que nous pourront ensuite détailler pour la ville de Rennes.
        
    \section{Base de valeurs}
    	Le but de la base de valeurs serait de fournir d'aider l'utilisateur à valuer ses noeuds, et lui permettre d'enregistrer des valeurs qui reviennent souvent.
      
    \section{Guide interactif}
    	Le guide doit être capable d'expliquer à l'utilisateur comment faire son analyse. 
        
        Il pourra par exemple poser une série de questions à l'utilisateur, afin de générer un arbre dit "de base" sur lequel l'utilisateur pourra commencer son analyse.
        
        Chaque notion devra être expliquée de manière claire et concise.
        
        Découper l'analyse en différentes étapes.        
        
    \section{\'Editeur d'arbres}
    	L'éditeur d'arbre sera l'outil AD Tool\footnote{Développé par...}, auquel nous rajouterons quelques fonctionnalités \textbf{(lesquelles?)} pour rendre l'édition des arbres plus souple.
    
    
	\section{Multiplateforme}
        Disscuss: Intérêt du multiplateforme.

    	Pro:
        \begin{itemize}
            \item Ne limite pas l'utilisateur dans son choix d'OS.
            \item AD Tool est déjà multiplateforme.
        \end{itemize}

        Cons:
        \begin{itemize}
        	\item Plus technique à réaliser.
        \end{itemize}

    \section{License}
    	GPL? MIT? BSD? WTF?
        
        Déterminer ce que l'on veut pour notre projet.
        
        Vérifier aussi license de AD Tool.