\chapter{Contexte}
	Le bon fonctionement des transports en commun dépends de nombreux facteurs. Que ce soit des facteurs humains, ou bien technique, le moindre dysfonctionnement impactera le systéme entier très rapidement. Ainsi, un simple tour d'horizon de la presse internationale fait remonter très vite à la surface de nombreux cas de paralysie des transports publiques urbain dans le monde entier.

	Parmi tout ces cas de paralysie, les plus marquant à l'échelle internationale sont ceux impliquant des dégats humains parmis les passagers. Il s'agit dans la plupart des cas d'attaque terroriste. Bien que relativement rare, le lourd bilan humain de ces attaques marque durablement les esprits.

	C'est le cas de l'attentat à la bombe dans la gare Saint-Michel du train inter-urbain parisien, le 25 juillet 1995 qui provoqua la mort de 8 personnes. Mais aussi 7 juillet 2005

	En France, il arrive aussi que les réseaux de transports en commun soient mis en difficulté. La plupart du temps, ces troubles sont dûs à des grèves du personnel pouvant refléter différentes réclamations. Ainsi, à Marseille en décembre 2013, les chauffeurs de bus protestent contre les salaires, la pénibilité en fin de carrière et la suppression de deux jours de congés. Cela est également arrivé à Lille en mai 2014, où le tramway et les bus sont bloqués par les traminotsqui demandent une hausse des salaires.

	Mais parfois, les grèves sont la conséquence de certains incidents survenus lors des trajets : le plus souvent, il s'agit d'agressions sur le personnel. Ces dernières sont plus fréquentes qu'on ne pourrait le penser.

	À Dunkerque en mai 2013 par exemple, un chauffeur subit une agression de la part de voyageurs, qui après leur exclusion du véhicule l'ont poursuivi en voiture jusqu'au terminus de la ligne. Arrivés là, équipés d'extincteurs, ils ont bloqué le bus et menacé ses occupants. La CGT a été avertie, une plainte a été déposée et les conducteurs ont exercé leur droit de retrait, paralysant ainsi le réseau de bus pendant toute une journée.

	C'est ensuite à Douai, en septembre de la même année, que trois contrôleurs sont agressés lors d'un contrôle par une vingtaine de personnes. Des coups sont échangés, les trois hommes finissent à l'hôpital avec des contusions et une entorse au poignet pour l'un d'entre eux. L'ensemble des contrôleurs du réseau exerce alors son droit de retrait, paralysant celui-ci pendant une journée entière...

	Mais qu'en est-il au sein de l'agglomération rennaise, qui nous intéresse tout particulièrement dans le cadre de ce projet ? Commençons par décrire le réseau de transports actuellement en place, ainsi que sa gestion.

	% * <corentin.nicole@insa-rennes.fr> 2014-10-06T12:49:44.420Z:
	%
	% --> le réseau de transport de Rennes est composé de...
	% Le "paragraph" est une proposition...

	\paragraph{À Rennes,} il existe une ligne de métro (une deuxième est en construction) et un réseau de bus, les deux étant gérés par un organisme nommé le STAR (Service des Transports en commun de l'Agglomération Rennaise). Ce dernier a également mis en place depuis quelques années un système de vélos en libre-service, les vélos STAR. Les usagers peuvent accéder aux différents services du STAR par plusieurs moyens : en achetant des tickets à l'unité (pour bus et métro), en utilisant une carte d'abonné rechargeable (la carte Korrigo)... Dans certaines stations de vélos STAR, ils peuvent même payer directement par carte bancaire. L'information aux voyageurs passe par 870 écrans dans les bus, 70 dans les stations de métro et 50 bornes d’informations voyageurs (BIV) dans les abribus. Le système d’aide à l’exploitation et à l’information des voyageurs (SAEIV) permet d’indiquer en temps réel le passage du prochain bus, les perturbations, les correspondances, la disponibilité des Vélos STAR… Ces données sont disponibles en open data, et consultables via un service mobile mis à disposition par le STAR.

	Toute cette organisation n'est cependant pas à l'abri des incidents et présente quelques failles : voici un récapitulatif des paralysies les plus importantes que nous avons trouvées.

	En juillet 2009, la ligne de métro est entièrement bloquée pendant près de 20h à la suite d'un violent orage provoquant l'inondation des voies de circulation. Ceci n'est certes pas une attaque volontaire mais cela reste une faiblesse du système qu'il nous a paru intéressant de relever. 

	C'est ensuite en avril 2012 que le réseau STAR entier est paralysé, en pleine heure de pointe, suite à l'agression d'un chauffeur. À l'époque, la direction recense 18 agressions depuis le début de l'année, et promet un redéploiement de ses agents de médiation et de prévention. Une mesure insuffisante pour la CFDT, qui réclame "une police dédiée aux transports".

	Environ un mois plus tard, en mai, la ligne de métro est bloquée dès 5h30 du matin. Selon le STAR, il s’agirait d’un problème informatique entre le centre de commandement du métro et les rames. En clair, la liaison qui permet de contrôler les rames à distance ne fonctionne plus. Jamais un tel incident ne s’était produit. Un service de bus a été mis en place pour limiter les conséquences.

	Ces quelques faits montrent bien l'importance de la mise en place d'une évaluation des risques, ainsi que d'un répertoire des défenses à utiliser pour contrer ces derniers.