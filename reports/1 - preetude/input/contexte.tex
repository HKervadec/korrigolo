\section{La sécurité des transports en commun}
    \subsection{Un contexte mondial}
        Le bon fonctionnement des transports en commun dépend de nombreux facteurs. Qu'il soit humain, ou bien technique, le moindre dysfonctionnement impactera le système entier très rapidement. Ainsi, un simple tour d'horizon de la presse internationale fait vite remonter à la surface de nombreux cas de paralysie des transports publics urbains dans le monde entier. Nous avons choisi d'évoquer ci-dessous les cas les plus marquants, ainsi que les plus fréquents, parmi ceux que nous avons trouvés. 
             
        Parmi tous les cas de paralysie, les plus marquants à l'échelle internationale sont ceux impliquant des dégâts humains importants parmi les passagers. Il s'agit, dans la plupart des cas, d'attaques terroristes. Bien que relativement rares, le lourd bilan humain de ces attentats marque durablement les esprits.  C'est le cas de l'attentat à la bombe dans la gare Saint-Michel du train inter-urbain parisien, le 25 juillet 1995, qui provoqua la mort de 8 personnes~\cite{stmichel}. L'attaque la plus marquante de l'histoire fut celle du 7 juillet 2005 dans la ville de Londres. Quatre attentats-suicides dans différentes rames de métro provoquent la mort de cinquante-six personnes et en blessent sept cents autres~\cite{london_attacks}.
        
        Si ces attaques sont impressionnantes et marquantes pour la population, elles restent minoritaires. Parmi les cas de paralysie que nous avons recensés, nous pouvons constater que la majorité des cas de blocage des transports en commun sont provoqués par le personnel des compagnies gérant ces transports. En effet, lors de conflits sociaux, le personnel possède un énorme moyen de pression sur la hiérarchie car il peut bloquer l'intégralité du trafic en se mettant en grève.  Ce fut le cas à Londres, où un plan de fermeture des guichets du métro a provoqué une grève massive le 28 avril 2014. Ce mouvement social a entraîné la fermeture des deux tiers des stations et la diminution du trafic de 50\%~\cite{tubeApril}. Ce fut aussi le cas à San Francisco, en juillet et octobre 2013, où une grève du réseau ferré inter-urbain (fort de ses quatre-cent-mille passagers par jour) a paralysé la ville pendant plusieurs jours~\cite{SFbart}. Des exemples peuvent aussi être cités en France, comme à Marseille en décembre 2013, où les conducteurs ont réussi à bloquer la quasi-totalité du réseau pendant plus de deux jours. Ils protestaient contre les faibles salaires, la pénibilité en fin de carrière et la suppression de deux jours de congés~\cite{greve_marseille}. Plus marquant, à Brest en juillet 2014, le tram est attaqué dans la  nuit par une douzaine de personnes à coups de cocktails molotov et de jets de pierres. Bien que personne n'ait été blessé, l'incident a fortement choqué la population brestoise.~\cite{molotov}.
            
    	L'informatisation des systèmes de transports crée une vulnérabilité suplémentaire. Les cartes d'abonnement des voyageurs peuvent être détournées, dans le but de récuperer des informations personnelles ou de voyager gratuitement. Par exemple, en 2008, des étudiants de l'université d'Amsterdam ont réussi à percer la sécurité d'un nouveau système de carte mise en projet par le gouvernement néerlandais.~\cite{Amst_RFID}. Les serveurs informatiques sont aussi vulnérables. Fin 2008, les serveurs du service de transport bruxellois, la STIB, sont victimes d'une attaque informatique~\cite{STIB}. 
    	Cette attaque fut vite repoussée par les ingénieurs en informatique de la société. Comme de nombreux services vitaux au bon fonctionnement du réseau de transports (cartes d'abonnement, coordination des bus, etc) dépendent de son système d'information, une attaque plus importante contre ce dernier pourrait provoquer de graves dysfonctionnements.

    	Pour illustrer notre projet, nous avons choisi de nous appuyer sur l'étude du réseau de transports publics de Rennes Métropole. Nous allons donc maintenant le décrire succinctement, afin de mieux comprendre par la suite le cadre de l'étude ainsi que les termes utilisés.
                
    \subsection{Le cas rennais}
    	Le réseau de transports publics rennais est constitué d'une ligne de métro automatique (une deuxième étant en construction) et d'un réseau de bus. Ces deux éléments sont gérés par le Service des Transports en commun de l'Agglomération Rennaise (STAR), qui dépend de la société Keolis Rennes. Le STAR a également mis en place depuis quelques années un système de vélos en libre-service : les vélos STAR. Les usagers peuvent accéder aux différents services du STAR par plusieurs moyens : en achetant des tickets à l'unité (pour bus et métro), ou en utilisant une carte d'abonné rechargeable (la carte Korrigo). 

        Les bus sont équipés de huit cent soixante-dix écrans, diffusant entre autres le nom des prochains arrêts du bus, les stations de vélos STAR à proximité, mais aussi quelques publicités. Des informations similaires sont visibles sur soixante-dix écrans dans les stations de métro. Les utilisateurs ont également à leur disposition cinquante bornes d’information aux voyageurs dans les abribus. Un système d’aide à l’exploitation et à l’information des voyageurs permet d’indiquer en temps réel le passage du prochain bus, les perturbations, les correspondances, la disponibilité des vélos STAR, etc~\cite{chiffres_star}. Toute cette organisation n'est cependant pas à l'abri des incidents, et présente quelques failles : voici un récapitulatif des exemples les plus importants que nous avons trouvés dans la presse.
        
        C'est en 2012 que les premières attaques volontaires sont relevées : en avril, le réseau STAR entier a été paralysé en pleine heure de pointe, suite à l'agression d'un chauffeur. À l'époque, la direction recensait dix-huit agressions depuis le début de l'année, et promettait un redéploiement de ses agents de médiation et de prévention. Une mesure insuffisante pour la Confédération Française Démocratique du Travail (CFDT), qui réclamait \og une police dédiée aux transports \fg ~\cite{bus_agression}. Environ un mois plus tard, en mai, la ligne de métro a été entièrement bloquée (dès 5h30) pendant toute une matinée. Des bus relais ont été mis en place pour desservir les stations, jusqu'à ce que la circulation du métro reprenne (vers 14h). Il s’agissait d’un problème informatique entre le centre de commandement du métro et les rames : la liaison qui permet de contrôler les rames à distance ne fonctionnait plus~\cite{metro_info}. 
        
        La ligne de métro a été paralysée en mai 2006 durant quatre heures, par deux chiens errants sur les voies. Le STAR ne s'explique pas comment les chiens ont pu pénétrer dans le tunnel~\cite{chiens_metro}. Dans la même catégorie d'attaques non volontaires, en juillet 2009, la ligne de métro a été bloquée pendant près de vingt heures à la suite d'un violent orage ayant provoqué l'inondation des voies de circulation~\cite{metro_orage}.
        
        Depuis 2012, des incidents se produisent régulièrement sur le réseau mais ne conduisent pas à l'arrêt du système. Il s'agit principalement d'agressions du personnel, comme ce fut le cas le 5 octobre 2014 où un contrôleur a reçu un coup de poing pendant l'exercice de ses fonctions~\cite{coup_poing_rennes}. Ces quelques faits montrent bien l'importance pour le STAR de la mise en place d'un outil d'évaluation des risques.
