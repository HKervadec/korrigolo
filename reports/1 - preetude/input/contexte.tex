\section{La sécurité des transports en commun}
    \subsection{Un contexte mondial}
        Le bon fonctionnement des transports en commun dépend de nombreux facteurs. Qu'il soit humain, ou bien technique, le moindre dysfonctionnement impactera le système entier très rapidement. Ainsi, un simple tour d'horizon de la presse internationale fait vite remonter à la surface de nombreux cas de paralysie des transports publics urbains dans le monde entier. 
             
        Parmi tous les cas de paralysie, les plus marquants à l'échelle internationale sont ceux impliquant des dégâts humains importants parmi les passagers. Il s'agit, dans la plupart des cas, d'attaques terroristes. Bien qu'ils soient relativement rares, le lourd bilan humain de ces attentats marque durablement les esprits.  C'est le cas de l'attentat à la bombe dans la gare Saint-Michel du train inter-urbain parisien, le 25 juillet 1995, qui provoqua la mort de 8 personnes~\cite{stmichel}. L'attaque la plus marquante de l'histoire fut celle du 7 juillet 2005 dans la ville de Londres. Quatre attentats-suicides dans différentes rames de métro provoquent la mort de 56 personnes et en blessent 700 autres~\cite{london_attacks}.
        
        Si ces attaques sont impressionnantes et marquantes pour la population, elles restent minoritaires dans les cas de paralysie. Parmi les cas de paralysie que nous avons recensé, nous pouvons constater que la majorité des cas de blocage des transports en commun sont provoqués par le personnel des compagnies gérant ces transports. En effet, lors de conflits sociaux, le personnel possède un énorme moyen de pression sur la hiérarchie car il peut bloquer l'intégralité du trafic en se mettant en grève.  C'est le cas à Londres, où un plan de fermeture des guichets du métro a provoqué une grève massive le 28 avril 2014. Cela a entraîné la fermeture des deux tiers des stations et la diminution du trafic de 50\%~\cite{tubeApril}. Mais aussi à San Francisco, en juillet et octobre 2013, où une grève du réseau ferré interurbain (forte de ses quatre-cent-mille passagers par jour) a paralysé la ville pendant plusieurs jours~\cite{SFbart}. Des exemples peuvent aussi être cités en France, comme à Marseille en décembre 2013, où les conducteurs ont réussi à bloquer la quasi-totalité du réseau pendant plus de deux jours. Ils protestaient contre les faibles salaires, la pénibilité en fin de carrière et la suppression de deux jours de congés~\cite{greve_marseille}. Plus marquant, à Brest en juillet 2014, le tram est attaqué dans la  nuit par une douzaine de personnes à coups de cocktails molotov et de jets de pierres. Si personne n'a été blessé, la conductrice en état de choc a été mise en arrêt et un sentiment d'incompréhension plane sur la ville~\cite{molotov}.
            
	L'informatisation des systèmes de transports créent une vulnérabilité suplémentaire. Les cartes d'abonnement des voyageurs peuvent être détournées, dans le but de récuperer des informations personnelles ou de voyager gratuitement. Par exemple, en 2008, des étudiants de l'université d'Amsterdam arrivent à percer la sécurité d'un nouveau système de carte mise en projet par le gouvernement néerlandais.~\cite{Amst_RFID}. Les serveurs informatiques sont aussi vulnérables. Fin 2008, les serveurs du service de transport bruxellois, la STIB, est victime d'une attaque informatique~\cite{STIB}. Si cette attaque fut vite repoussé par les ingénieurs informatique de la société, une attaque plus importante pourrait provoquer de grave dysfonctionnement du systéme d'information du réseau de transport. Or de nombreux services vitaux au bon fonctionnement du réseau de transport dépendent ce systéme d'information : les cartes d'abonnement, la coordination des bus, etc...

	Pour illustrer notre projet, nous avons choisi de nous appuyer sur l'étude du réseau de transports publics de Rennes Métropole. Nous allons donc maintenant le décrire succinctement, afin de mieux comprendre par la suite le cadre de l'étude ainsi que les termes utilisés.
                
\subsubsection{Le cas rennais}
	Le réseau de transports publics rennais est constitué d'une ligne de métro (une deuxième étant en construction) et d'un réseau de bus. Ces deux éléments sont gérés par le Service des Transports en commun de l'Agglomération Rennaise (STAR), qui dépend de la société Keolis Rennes. Le STAR a également mis en place depuis quelques années un système de vélos en libre-service : les vélos STAR. Les usagers peuvent accéder aux différents services du STAR par plusieurs moyens : en achetant des tickets à l'unité (pour bus et métro), ou en utilisant une carte d'abonné rechargeable (la carte Korrigo). 
                


                %REF pour les numbres !?
        L'information aux voyageurs passe par 870 écrans dans les bus, 70 dans les stations de métro et 50 bornes d’informations voyageurs (BIV) dans les abribus. Le système d’aide à l’exploitation et à l’information des voyageurs (SAEIV) permet d’indiquer en temps réel le passage du prochain bus, les perturbations, les correspondances, la disponibilité des vélos STAR… 
        % pas de 3 ptits points !
        Ces données sont disponibles en open data, et consultables via un service mobile mis à disposition par le STAR. Toute cette organisation n'est cependant pas à l'abri des incidents et présente quelques failles : voici un récapitulatif des paralysies les plus importantes que nous avons trouvées. 
        
        L'unique ligne de métro de Rennes a été paralysée en mai 2006 durant quatre heures par deux chiens errants sur les voies. Le STAR ne s'explique pas comment les chiens ont pu pénétrer dans le tunnel. Dans la même catégorie des attaques non volontaires, en juillet 2009, la ligne de métro a été bloquée pendant près de vingt heures à la suite d'un violent orage ayant provoqué l'inondation des voies de circulation~\cite{metro_orage}.
        
        C'est ensuite en 2012 que les premières attaques volontaires sont relevées : en avril, le réseau STAR entier a été paralysé en pleine heure de pointe, suite à l'agression d'un chauffeur. À l'époque, la direction recensait 18 agressions depuis le début de l'année, et promettait un redéploiement de ses agents de médiation et de prévention. Une mesure insuffisante pour la Confédération Française Démocratique du Travail (CFDT), qui réclamait "une police dédiée aux transports"~\cite{bus_agression}. Environ un mois plus tard, en mai, la ligne de métro a été entièrement bloquée (dès 5h30) pendant toute une matinée. Des bus relais ont été mis en place pour desservir les stations, jusqu'à ce que la circulation du métro reprenne (vers 14h). Selon le STAR, il s’agirait d’un problème informatique entre le centre de commandement du métro et les rames : la liaison qui permet de contrôler les rames à distance ne fonctionnait plus~\cite{metro_info}. 
        
        Depuis 2012, des incidents se produisent régulièrement sur le réseau mais ne conduisent pas à l'arrêt du système. Il s'agit principalement d'agressions du personnel, comme ce fut le cas le 5 octobre 2014 où un contrôleur a reçu un coup de poing pendant l'exercice de ses fonctions~\cite{coup_poing_rennes}. Ces quelques faits montrent bien l'importance pour le STAR de la mise en place d'un outil d'évaluation des risques, qui comporterait également un répertoire de défenses à utiliser pour contrer ces derniers.
