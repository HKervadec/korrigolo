\section{Risques}
		Afin d'améliorer la gestion de notre projet, et de ne pas nous faire surprendre par des évènements imprévus, nous avons décidé de les recenser. Cette liste donne les principaux problèmes potentiels ainsi que les éléments de résolution envisagés.
		
    \subsection{Risques humains}
		Le risque pouvant impacter le plus notre projet est de ne pas respecter le cahier des charges. Le facteur principal est humain : en effet, trois membres du groupe partent étudier à l'étranger dans le cadre de la mobilité internationale. Il n'en restera donc que trois à travailler sur l'implémentation. De ce fait, un investissement moindre dans le projet de l'un des membres restants conduirait à un retard d'autant plus important au second semestre. Planifier une réunion hebdomadaire et compter nos heures devrait permettre d'éviter ce problème. Il se peut également que nous ayons légèrement sur-estimé nos compétences, notre capacité à nous former, et annoncé une tâche trop difficile à exécuter. L'application pourrait nécessiter plus de temps que nous ne l'avions initialement prévu, et donc que nous ne puissions pas rendre un projet respectant le cahier des charges initial.
		
    \subsection{Risques techniques}
	    Il existe aussi des risques plus facilement identifiables. Nous pouvons par exemple nous rendre compte que nos choix d'implémentation, annoncés dans la Section \ref{sec:cahier}, ne fonctionnent pas comme nous l'avions prévu. Il est aussi possible que nous rencontrions des difficultés à déployer l'application, la plupart d'entre nous n'en ayant jamais développé. Nous pourrions ainsi connaître des problèmes dans la communication des différents modules. Nous nous serons renseignés sur le sujet et aurons envisagé plusieurs solutions, mais si celle retenue ne donne pas entière satisfaction, nous serons obligé d'en changer. Si ceci intervient à un stade avancé du projet, il y aura un retard dû à l'adaptation de l'implémentation. Ceci peut aboutir à une impossibilité de livrer le logiciel dans l'état dans lequel nous le décrivons ici.

	    La perte des données est un risque majeur en informatique, et peut potentiellement faire perdre l'intégralité de plusieurs mois de travail. Toutefois, l'utilisation de Git et le fait que nous ayons chacun une copie (à jour) du dépôt sur nos ordinateurs supprime quasiment ce risque.

	\subsection{Critiques}

		Dès cette première livraison, nous pouvons identifier des pistes d'amélioration dans notre organisation interne, en particulier dans le cadre de la rédaction de ce rapport. En effet, sitôt le plan du rapport établi, nous avons réparti le travail de rédaction entre nous sans nous concerter de manière précise sur le contenu. L'homogénéité du rapport a souffert de ce cloisonnement. Nous avons donc perdu du temps pour résoudre ce problème. \`A l'avenir, nous pourrons apprendre de cette erreur due principalement à notre manque d'expérience dans ce domaine. En effet notre groupe de travail est plus important que d'habitude et les rapports plus conséquents.
