\section{Risques}
   % Nous pouvons d'ores et déjà envisager des situations dans lesquelles nous ne serions pas en mesure de livrer le projet dans l'état que nous avions prévu en ce début d'année scolaire.
    Pour une meilleure gestion de projet, nous avons tenu à envisager les divers risques liés à notre projet.

    \subsection{Risques humains}
		Le plus gros risque, et celui impactant le plus notre projet, est de ne pas suivre le cahier des charges, pour quelque raison que ce soit.
	    Le facteur principal est bien entendu humain. En effet, trois membres du groupe partant étudier à l'étranger dans le cadre de la mobilité internationale, il en restera trois à travailler sur le projet. Il se peut que nous ayons légèrement sur-estimé nos compétences, notre capacité à nous former, et annoncé une tâche trop difficile à exécuter. De plus, un investissement moindre dans le projet de l'un des membres restant, pour une raison quelconque, conduirait à un retard d'autant plus important au second semestre. Pour éviter une telle situation, nous avons planifié une réunion hebdomadaire où nous mettons en commun nos idées et nos avancées. Toujours dans cette optique, sans être trop compliquée, l'application pourrait nécessiter plus de temps que nous l'avions initialement prévu, et donc que nous ne puissions pas rendre un projet respectant le cahier des charges initial.

    \subsection{Risques techniques}
	    Toujours dans cette optique, nous pouvons également nous rendre compte que nos choix d'implémentation, annoncés dans la Section \ref{sec:cahier} et qui sont pour le moment théoriques, ne fonctionnent pas comme nous le voudrions. Par exemple, l'application mère pourrait avoir besoin d'un module supplémentaire qui servirait à effectuer la conversion de fichier à importer dans ADTool. Nous avons annoncé une réunion, au cours de laquelle nous déciderions du langage à utiliser pour coder notre application.
	    Il est aussi possible que nous rencontrions des difficultés à déployer l'application, la plupart d'entre nous n'en ayant jamais développé. Nous pourrions ainsi connaître des problèmes dans la communication des différents modules. Nous nous serons renseignés sur le sujet et aurons envisagé plusieurs solutions, mais si celle retenue ne donne pas entière satisfaction, nous serons obligé d'en changer. Si ceci intervient à un stade avancé du projet, il y aura un retard dû à l'adaptation de l'implémentation, qui peut aboutir à une impossibilité de livrer l'application dans l'état dans lequel nous la décrivons ici.

	    La perte des données est un risque majeur en informatique, et peut potentiellement faire perdre l'intégralité de plusieurs mois de travail. Toutefois, l'utilisation de Git et le fait que nous ayons chacun une copie (à jour) du repository sur nos ordinateurs supprime presque complètement ce risque.

	\subsection{Critiques}
		Dès cette première livraison, nous avons identifié des points d'amélioration de notre organisation interne. Nos encadrants nous ont aidé à résoudre ce problème, et nous ont guidé afin d'établir dans le groupe une structure claire et une répartition simple des rôles.
		Nous avons également eu des difficultés d'organisation pour la rédaction de ce rapport. Dès le début, nous avons réparti le travail de rédaction entre nous sans nous concerter de manière précise sur le contenu. La mise en commun a été rendue plus difficile par cette séparation. Nous pourrons apprendre de cette erreur afin de combler notre manque d'expérience dans ce domaine. En effet notre groupe de travail est plus important que d'habitude et les rapports seront plus nombreux et conséquents.