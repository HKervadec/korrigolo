\chapter{Risques}
	Nous pouvons d'ores et déjà envisager des situations dans lesquelles nous ne serions pas en mesure de livrer le projet dans l'état que nous avions prévu en ce début d'année scolaire.
	Le facteur principal est bien entendu humain. En effet, la moitié du groupe partant étudier à l'étranger dans le cadre de la mobilité internationale, il se peut que nous ayons légèrement sur-estimé nos capacités de travail et annoncé une tâche trop difficile à exécuter. Des soucis personnels causant l'ineficacité d'un membre du groupe (particulièrement un membre restant au second semestre) comprommetrait également l'avancée du projet.

	Des problèmes d'ordre technique peuvent également survenir. Des difficultés à mettre en place la chaîne logicielle prévue, des soucis de langage utilisé ou encore de compatibilité sont possibles.
	Une perte de données stockées sur Git, ou bien de fauses manipulations entrainant une perte de données totale ou partielle ralentirait \textit{a minima} le projet, et pire le rendrait impossible à livrer tel que nous l'avons promis.