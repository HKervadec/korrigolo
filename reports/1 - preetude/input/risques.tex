\section{Risques}
    Nous pouvons d'ores et déjà envisager des situations dans lesquelles nous ne serions pas en mesure de livrer le projet dans l'état que nous avions prévu en ce début d'année scolaire.

    Le plus gros risque, et celui impactant le plus notre projet, est de ne pas suivre le cahier des charges, pour quelque raison que ce soit.
    Le facteur principal est bien entendu humain. En effet, trois membres du groupe partant étudier à l'étranger dans le cadre de la mobilité internationale, il en restera la même nombre à travailler sur le projet. Il se peut que nous ayons légèrement sur-estimé nos compétences, notre capacité à nous former, et annoncé une tâche trop difficile à exécuter. De plus, un désintéressement du projet de la part de l'un des membres restant, pour une raison quelconque, conduirait à un retard d'autant plus important au second semestre. Pour éviter une telle situation, nous avons planifié une réunion hebdomadaire où nous mettons en commun nos idées et nos avancées. Toujours dans cette optique, sans être trop compliquée, l'application pourrait nécessiter plus de temps que nous l'avions initialement prévu, et donc que nous ne puissions pas rendre un projet respectant le cahier des charges initial.

    Toujours dans cette optique, nous pouvons également nous rendre compte que nos choix d'implémentation, annoncés dans la section \ref{sec:cahier} et qui sont pour le moment théoriques, ne fonctionnent pas comme nous le voudrions. Par exemple, l'application mère pourrait avoir besoin d'un module supplémentaire qui servirait à effectuer la conversion de fichier à importer dans ADTool. Nous avons annoncé une réunion, au cours de laquelle nous déciderions du langage à utiliser pour coder notre application.
    Ces problèmes sont plutôt d'ordre technique, et d'autres peuvent également survenir. Il est ainsi possible que nous rencontrions des difficultés à déployer l'application, la plupart d'entre nous n'en ayant jamais développé. Nous pourrions ainsi connaître des problèmes dans la communication des différents modules. Nous nous serons renseignés sur le sujet et aurons envisagé plusieurs solutions, mais si celle retenue ne donne pas entière satisfaction, nous serons obligé d'en changer. Si ceci intervient à un stade avancé du projet, il y aura un retard dû à l'adaptation de l'implémentation, qui peut aboutir à une impossibilité de livrer l'application dans l'état dans lequel nous la décrivons ici.

    Une perte de données stockées sur GitHub, ou bien de fausses manipulations entraînant une perte de données totale ou partielle ralentirait \textit{a minima} le projet, et dans le pire des cas le rendrait impossible à livrer tel que nous l'avons promis. Cependant, GitHub est équipé d'outils de récupération pour les \textit{commit}, et le projet sera régulièrement mis à jour (\textit{pull}) sur chacun de nos ordinateurs personnels, réduisant quasiment ce risque à zéro.

    % Approfondir risque:
    % Risque de pas suivre cahier charges