\section{Risques}
    Nous pouvons d'ores et déjà envisager des situations dans lesquelles nous ne serions pas en mesure de livrer le projet dans l'état que nous avions prévu en ce début d'année scolaire.
    Le facteur principal est bien entendu humain. En effet, la moitié du groupe partant étudier à l'étranger dans le cadre de la mobilité internationale, il se peut que nous ayons légèrement sur-estimé nos capacités de travail et annoncé une tâche trop difficile à exécuter. De plus, sur les trois personnes restant en France, une incapacité à travailler sur le projet, quelle qu'elle soit, aurait des répercussions sur l'avancée globale du livrable. Mais une incapacité de ce type serait accidentelle et très peu probable.
    
    Des problèmes d'ordre technique peuvent également survenir. Il est ainsi possible que nous rencontrions des difficultés à mettre en place la chaîne logicielle prévue, la plupart d'entre nous n'en ayant jamais développé. Nous pourrions ainsi connaître des problèmes dans la communication des différents composants de la chaîne logicielle. Il faudra, pour éviter cela, bien choisir les langages utilisés dans la conception logicielle.
    Une perte de données stockées sur Git, ou bien de fauses manipulations entrainant une perte de données totale ou partielle ralentirait \textit{a minima} le projet, et dans le pire des cas le rendrait impossible à livrer tel que nous l'avons promis. Cependant, Git est équipé d'outils de récupération pour les \textit{commit}, et le projet sera normalement mis à jour (\textit{pull}) sur chacun de nos ordinateurs personnels, réduisant quasimment ce risque à zéro.
