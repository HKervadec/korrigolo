\section{Introduction}
	Un long chemin a été parcouru depuis le 18 mars 1662, où le brillant penseur Blaise Pascal réalise la première expérience de transports en commun urbains au monde. Il s'agit alors de sept carrosses publics qui sont mis en service entre la Porte Saint-Antoine et le palais du Luxembourg à Paris. Depuis lors, les transports en commun ont beaucoup évolué, et sont aujourd'hui indispensables au fonctionnement d'une ville. En témoignent ces chiffres impressionnants : selon les estimations du Groupement des Autorités Responsables de Transport (GART), 6.5 millions de trajets sont réalisés par jour en France~\cite{Gart}. \`A Rennes, l'agglomération sur laquelle portera notre étude, la société Keolis dénombre une moyenne de 250 000 trajets par jour~\cite{Keolis}. 

	Nous comprenons donc aisément l'importance que revêt le bon fonctionnement de ces transports publics en métropole. Ils génèrent des retombées économiques, réduisent de manière significative les embouteillages urbains et participent à l’amélioration de la qualité de vie de la population. Par exemple, à Rennes, la ligne A du métro permet d'économiser 11 000 tonnes de CO2 par an~\cite{bilanLA}. 

	Une paralysie de ces derniers aurait une incidence considérable sur l'ensemble de la métropole. En découle la nécessité de réaliser une étude de sécurité pour prévoir les risques les plus importants et mettre au point une défense efficace. 

























