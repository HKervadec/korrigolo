\section{Introduction}

\subsection{Présentation du sujet}


Un long chemin a été parcouru depuis le 18 mars 1662, où le brillant penseur Blaise Pascal réalise la première expérience de transports en commun urbains au monde. Il s'agit alors de sept carrosses publics qui sont mis en service entre la Porte Saint-Antoine et le palais du Luxembourg à Paris. Depuis lors, les transports en commun ont beaucoup évolué. Ils sont aujourd'hui indispensables au fonctionnement d'une ville. En témoignent ces chiffres impressionnants : selon les estimations du Groupement des Autorités Responsables de Transport (GART), 6.5 millions de trajets sont réalisés par jour en France. \'A Rennes, l'agglomération sur laquelle se portera notre étude, la société Keolis dénombre une moyenne de 250 000 trajets par jour. %REF


Nous comprenons donc aisément l'importance que revêt le bon fonctionnement de ces transports publics en métropole. Ils génèrent des retombées économiques, réduisent de manière significatifs les embouteillages urbains et participent à l’amélioration de la qualité de vie de la population. A Rennes, la ligne A du métro permet d'économiser 11 000 T de CO2 par an \cite{bilanLA}

 
Une paralysie de ces derniers aura une incidence considérable sur l'ensemble de la métropole.

\subsection{Les objectifs du projet}

En se basant sur le concept des arbres d’attaque et de défense (ADTree), nous allons développer une application destinée à faciliter la tâche des experts en sécurité. Nous partons du principe que les utilisateurs de notre application maîtrisent les ADTree, ou du moins qu’ils n’y sont pas totalement étrangers.


L’étude des transports publics urbains rennais nous permettra d’illustrer notre projet. Cela nous servira d’exemple, mais notre application pourra être utilisée de manière plus générale. Elle pourra résoudre un certain nombre de problèmes de sécurité, en prenant en compte différents paramètres : profil d’attaquant, cible visée, etc. 

























