\section{Introduction}
    Un long chemin a été parcouru depuis le 18 mars 1662, où le brillant penseur Blaise Pascal réalise la première expérience de transports en commun urbains au monde. Il s'agit alors de sept carrosses publiques qui sont mis en service entre la Porte Saint-Antoine et le palais du Luxembourg à Paris. Depuis lors, les transports en commun ont beaucoup évolué. Ils sont aujourd'hui indispensable au fonctionnement d'une ville. En témoigne ces chiffres impressionants : selon les estimations du Gart, 6.5 million de trajet sont réaliséés par jours en France.A Rennes, l'agglomération sur lequel se portera notre étude, la société Keolis dénombre une moyenne de 250 000 trajets par jour. 

    Nous comprenons donc aisément, l'importance que revêt le bon fonctionemment de ces transports en commun. Une paralysie de ces derniers aura une incidence considérable l'ensemble de la métropole. 

    De plus la concentration de la population dans les transports implique un dommage humain. conséquent en cas d'attaque létale.


    Cette étude consistera à se placer en tant qu'attaquant, pour essayer de trouver les failles qui permettraient à une personne mal intentionnée de paralyser les transports en commun de Rennes métropole. Dans ce but nous utiliserons la théorie des arbres d'attaques de Bruce Schneier. 

    En paralléle de cette étude nous réaliserons une suite logicielle. Celle-ci aura pour finalité d'assister les experts en sécurité.