\section{Organisation et planification}
	\subsection{Organisation}
	    Etant donné que trois membres du groupe partent à l'étranger au second semestre, nous avons fixé une organisation précise pour que le projet se déroule de la meilleure manière possible. Il a ainsi été décidé que les différents rôles au sein du groupe seraient les suivants :
	    \paragraph{Coordinateur} Le coordinateur s'occupe de planifier les réunions de projet, de les animer et est le contact des encadrants. Son rôle est aussi de s'assurer de l'avancée des rapports et de leur relecture. Il changera régulièrement afin que tout le groupe assure ce rôle. Hoël Kervadec est le premier coordinateur, jusqu'à la livraison de ce rapport. Corentin Nicole sera le prochain coordinateur, puis Florent Mallard assurera ce rôle. De cette manière, chacun des étudiants partant en études à l'étranger aura tenu la place de coordinateur avant son départ.
	    \paragraph{Sysadmin} L'administrateur système s'occupe de maintenir à jour les plate-formes et outils utilisés durant le projet (GitHub, GoogleDrive). Valentin Esmieu est en charge de ces plate-formes.
	    \paragraph{Scribe} Un scribe est volontaire à chaque début de réunion afin de prendre les notes. Ce rôle est communément assuré par les personnes disposant de leur ordinateur à l'INSA afin de rédiger un compte-rendu en direct sur le GoogleDrive.
	    \paragraph{Responsable Planification} Le responsable planification est en charge du suivi et de la mise à jour chaque semaine de planification du projet. Celui-ci utilisera le logiciel MS Project comme indiqué dans la section \ref{sec:outils}.
	    \paragraph{Décompte du temps} Au sein du groupe, le décompte du temps passé sur le projet se fait de manière autonome, chacun indique le temps passé sur telle ou telle tâche. Nous utilisons pour cela un tableur créé sur le GoogleDrive du projet. Le coordinateur et le responsable planification consulteront ce tableur afin d'effectuer au mieux la répartition des tâches restantes.

	    Nous avons également décidé de nous réunir de manière hebdomadaire, le mercredi soir. Au cours de celle-ci, nous nous concertons sur les tâches à réaliser, et nous les répartissons entre nous. C'est aussi à ce moment que nous débattons sur les principales questions qui seront posées au cours de la prochaine réunion avec nos encadrants.

	    
	\subsection{Planification}


		Nous nous sommes familiarisés avec ADTool, fourni par notre encadrante, et nous avons suivi des cours sur les ADTrees pendant les trois premières séances de projet. Le logiciel ADTool permet de créer et de valuer selon certains critères un arbre d'attaque. C'est dans ce cadre que nous avons aperçus des points d'amélioration tels que l'intégration de l'importation d'arbres annoncée dans la section \ref{sec:cahier}. Nous implémenterons ces différentes fonctionnalités afin de rendre l'utilisation d'ADTool plus complète pour notre projet.
	    
	    Après cette introduction à ADTool, nous avons jugé utile de suivre un cours de cryptographie, afin de mieux saisir les protocoles de communication sécurisée. Gildas Avoine nous a donc dispensé ce cours durant deux heures de notre temps libre. Cela nous a permis d'appréhender les concepts de protection des cartes Korrigo. Ils pourraient nous être utiles pour analyser et valuer les risques concernant ce type d'attaque.
			
		Si la planification n'est pas encore totalement établie, nous pouvons déjà en donner les principaux axes.

		Étant donné que nous avons noté le temps passé à travailler sur ce rapport, nous pouvons en déduire approximativement le temps qu'il nous faudra afin de livrer les suivants. Pour la livraison d'un rapport de vingt pages, nous passons, entre rédaction et correction, environ quinze heures chacun.

		Ainsi, à partir de ce jour et jusqu'à la mi-novembre, nous allons définir les spécifications fonctionnelles de notre projet.
		Pour ce faire, nous allons donc préciser la Figure \ref{fig:archi} afin de mettre en lumière les différentes interactions des modules présents, et la nature de celles-ci. Nous nous réunirons également pour décider des technologies à utiliser pour l'implémentation de notre application, et donnerons une idée précise de son interface graphique et des différents moyens d'interaction présents.
		Une fois cette décision prise, nous nous mettrons d'accord sur la manière d'implémenter les améliorations de l'outil ADTool et sur la répartition des tâches.

		Pour cela, jusqu'aux vacances de Noël, nous établirons la planification complète du projet. Elle sera plus individualisée et détaillée, et contiendra un diagramme de Gantt expliquant la répartition et la simultanéité des différentes tâches qui seront entreprises.
		A ce moment nous aurons donc tous les outils nécessaires à la modélisation de notre projet.
		Au terme de cette échéance nous serons en mesure de présenter notre planification ainsi que nos spécifications lors d'une session orale.

		De début 2015 à mi-février, nous définirons donc l'architecture interne de notre logiciel, et nous pourrons en parallèle commencer l'implémentation.
		Suivra enfin l'implémentation complète de notre application, à partir du début de l'année 2015 jusqu'à la fin de l'année scolaire.
		Nous présenterons enfin le projet complet durant une dernière session orale.

	    
