\section{Planification et organisation}
    Durant nos six premières heures de projet, nous avons beaucoup discuté de notre organisation avec nos encadrants. Il a ainsi été décidé que les rôles seraient les suivants :
    \paragraph{Coordinateur} Le coordinateur s'occupe de planifier les réunions de projet, de les animer et est le contact des encadrants. Son rôle est aussi de s'assurer de l'avancée des rapports et de leur relecture. Il changera régulièrement afin que tout le groupe assure ce rôle. Hoel Kervadec est le premier coordinateur, jusqu'à la livraison de ce rapport.
    \paragraph{Sysadmin} L'administrateur système s'occupe de maintenir à jour les plate-formes et outils utilisés durant le projet (GitHub, GoogleDrive). Valentin Esmieu est en charge de ces plate-formes.
    \paragraph{Scribe} Un scribe est volontaire à chaque début de réunion afin de prendre les notes. Ce rôle est communément assuré par les personnes disposant de leur ordinateur à l'INSA afin de rédiger un compte-rendu en direct sur le GoogleDrive.
    \paragraph{Responsable Planification} Le responsable planification est en charge du suivi et de la mise à jour chaque semaine de planification du projet. Celui-ci utilisera le logiciel MS Project comme indiqué dans la section "Outils"'.
    \paragraph{Décompte du temps} Au sein du groupe, le décompte du temps passé sur le projet se fait de manière autonome, chacun indique le temps passé sur telle ou telle tâche. Nous utilisons pour cela un tableur créé sur le GoogleDrive du projet. Le coordinateur et le responsable planification consulteront ce tableur afin d'effectuer au mieux la répartition des tâches restantes.

    Nous nous sommes également familiarisés avec l'outil ADTool, l'outil fourni par notre encadrante, et avons suivi des cours sur la théorie des arbres d'attaque et de défense pendant les trois premières séances de projet. Nous avons aperçus des points d'amélioration que nous serons peut-être amenés à implémenter afin de rendre son utilisation plus complète.
    
    Après cette introduction à ADTool, nous avons jugé utile un cours de cryptographie, afin de mieux saisir les protocoles de communication sécurisée. Gildas Avoine nous a donc dispensé ce cours durant deux heures de notre temps libre. Cela nous a permis d'appréhender les concepts de protection des cartes Korrigo, qui nous serons sûrement amenés à analyser dans le cadre d'une paralysie de la STAR.