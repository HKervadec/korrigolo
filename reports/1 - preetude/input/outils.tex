\section{Outils}
    Afin de développer efficacement notre projet, nous allons utiliser les outils suivants.
     
    \subsection{Gestion de projet}
        Pour permettre le versionnement de notre projet, nous avons choisi d'utiliser {\bf Git} pour son efficacité et sa simplicité d'utilisation. De plus, il semble s'imposer progressivement dans le monde professionnel, au détriment de Subversion.
        
        Pour le partage de fichiers lourds, la rédaction de notes de réunion, ou plus simplement pour mettre rapidement des idées en commun, nous avons décider d'utiliser {\bf Google Drive}, que nous préférons à Dropbox, car ce dernier ne permet pas d'écrire simultanément sur un même document.

        Enfin, en outil de plannification, {\bf MS Project} nous est imposé.

    \subsection{Langages de programmation et compatibilité entre systèmes}
        Nous comptons apporter quelques améliorations à ADTool, qui est écrit en {\bf Java}. Nous utiliserons donc probablement {\bf IntelliJ IDEA} comme environnement de développement Java pour réaliser ces améliorations.
        
        De par le fait qu'il soit écrit en Java, ADTool est {\itshape a priori} un logiciel multiplateforme, bien que n'ayant pas été testé sur Macintosh. Il pourrait être intéressant que notre suite logicielle reste compatible sur toutes les plateformes. Toutefois, nous ayons décidé de ne pas en faire notre priorité : nous nous focaliserons donc sur l'environnement {\bf Windows}, car c'est celui que nous maîtrisons le plus, et le plus susceptible d'être utilisé en entreprise.
        
        Nous souhaitons cependant ne pas volontairement empêcher toute compatibilité avec Linux et Macintosh : à ce titre, Java serait idéal pour développer le(s) logiciel(s) supplémentaire(s) de notre suite, mais nous ne souhaitons pas non plus ne nous restreindre qu'à Java tout au long de ce projet.
        
        Ainsi, l'emploi du {\bf C++} nous parait le meilleur compromis : le C++ devrait toujours permettre une compatibilité correcte sur Linux et Mac tant que nous utilisons des librairies standardisées, et c'est un langage extrêmement utilisé en entreprise dont nous ne voyons que les bases dans notre cursus. L'expérience que peut nous apporter ce projet nous conforte donc dans notre volonté de nous confronter à ce langage. Par conséquent, nous utiliserons {\bf Visual Studio} pour développer en C++, car c'est, à notre connaissance, l'outil le plus puissant et le plus utilisé actuellement pour ce langage.

    \subsection{Interfaces graphiques}
        L'interface graphique de notre suite est primordiale pour garantir aux utilisateurs novices un certain confort d'utilisation. Dans un souci de non-incompatibilité avec Linux et Macintosh, notre choix s'est porté sur {\bf Qt} pour l'interface que présenteront les logiciels de notre suite, autre que ADTool. 
        
        Pour ce dernier, nous considérons que le logiciel est déjà suffisamment ergonomique, et qu'y apporter des améliorations graphiques demanderait une charge de travail élevée pour une amélioration minime. Par conséquent, nous ne prévoyons pas de modification de son interface.