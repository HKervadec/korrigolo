\chapter{Outils}
	Voici une description des différents outils que nous allons utiliser pour développer notre projet.
	 
	\section{Outils collaboratifs}
		Afin de versionner efficacement notre projet, nous avons décidé d'utiliser {\bf Git} pour travailler comme des professionnels.

		Pour le partage de fichiers lourds et l'édition de comptes-rendus de réunion, {\bf GoogleDrive} s'est imposé comme le leader mondial de l'univers.

		Enfin, pour la plannification, {\bf MS Project} nous est imposé.

	\section{Langages de programmation}
		Etant donné que nous souhaitons réaliser une suite logicielle, la communication entre les différents programmes doit pouvoir se faire facilement.

		\begin{tabular}{llll}
		   	 					& {\bf Java} 	& {\bf C++} 	& {\bf C\#}  \\
		   {\bf Avantages} 		& azerty		& azerto		& truc\\
		   {\bf Inconvénients}  & azerty 		& zfdsfs		& dwfgdf\\
		\end{tabular}

		Après réflexion, on va prendre ce langage là. Par conséquent, on va utiliser {\bf Visual Studio} pour développer à l'aise.

	\section{Interfaces graphiques}
		Afin de garantir un confort d'utilisation optimal pour des utilisateurs novices, on va fait de belles interfaces.
		GUI :
		WinForm
		WMA
		GTK+
		Qt




