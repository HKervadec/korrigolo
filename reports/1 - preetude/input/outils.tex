\section{Outils}
    \label{sec:outils}
    	Nous souhaitons utiliser des outils efficaces et employés en milieu professionnel, afin de répondre au mieux à notre cahier des charges tout en en tirant une expérience intéressante. Voici donc les différents outils que nous allons employer.
         
    \subsection{Outils de gestion de projet}
        Pour permettre le versionnement de notre projet, nous avons choisi d'utiliser {\bf Git} \footnote{https://github.com} pour son efficacité et sa simplicité d'utilisation. De plus, il semble s'imposer progressivement dans le monde professionnel, au détriment de Subversion.
        
        Pour le partage de fichiers lourds, la rédaction de notes de réunion, ou plus simplement pour mettre rapidement des idées en commun, nous avons décidé d'utiliser {\bf Google Drive}, que nous préférons à Dropbox, car ce dernier ne permet pas d'écrire simultanément sur un même document.

        Enfin, il nous est demandé d'employer {\bf MS Project} comme outil de planification, que nous utiliserons donc pour tenir à jour le planning du projet.

	\subsection{Systèmes d'exploitation}
	   Comme mentionné précédemment, nous développerons notre application pour {\bf Windows}, en raison de sa très grande présence dans les entreprises. Nous souhaitons cependant garder la porte ouverte à une compatibilité avec Linux et Macintosh, au cas où un futur portage serait envisagé. Ces considérations nous ont amenés à choisir les langages de programmation décrits ci-après.
	
    \subsection{Langages de programmation et environnements de développement intégrés}
        Pour notre application, l'emploi du {\bf C++} nous parait pertinent. Bien que, de par sa nature, Java permette une compatibilité sur tous les systèmes d'exploitation, nous souhaitons éviter d'utiliser ce langage sur la totalité du projet. En effet, le C++ étant très répandu en entreprise, nous souhaitons nous y confronter. De plus, tant que nous utilisons des librairies standardisée, un portage sur Linux ou Macintosh reste possible. Par conséquent, nous utiliserons {\bf Visual Studio} pour développer en C++, car c'est l'environnement de développement intégré actuellement le plus puissant et le plus utilisé  pour ce langage.
        
        Enfin, pour les améliorations apportées à ADTool --- qui est écrit en Java --- nous emploierons {\bf IntelliJ IDEA}.

    \subsection{Interfaces graphiques}
        L'interface graphique de notre suite est primordiale pour garantir une certaine ergonomie à l'utilisateur. Toujours dans un souci de compatibilité, notre choix s'est porté sur {\bf Qt} pour l'interface que présenteront les logiciels de notre suite --- autre que ADTool. 
        
        Pour ce dernier, nous considérons que le logiciel est déjà suffisamment simple et intuitif, et qu'y apporter des améliorations graphiques demanderait une charge de travail élevée pour une amélioration minime. Par conséquent, nous ne prévoyons pas de modification de son interface.
