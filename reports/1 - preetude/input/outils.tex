\section{Outils}
    Le choix des technologies que nous allons employer pour ce projet est primordial. Nous souhaitons utiliser des outils efficaces et employés en milieu professionnel, afin de répondre au mieux à notre cahier des charges tout en en tirant une expérience intéressante. Voici donc les différents outils que nous souhaitons utiliser.
         
    \subsection{Outils de gestion de projet}
        Pour permettre le versionnement de notre projet, nous avons choisi d'utiliser {\bf Git} \footnote{https://github.com} pour son efficacité et sa simplicité d'utilisation. De plus, il semble s'imposer progressivement dans le monde professionnel, au détriment de Subversion.
        
        Pour le partage de fichiers lourds, la rédaction de notes de réunion, ou plus simplement pour mettre rapidement des idées en commun, nous avons décider d'utiliser {\bf Google Drive}, que nous préférons à Dropbox, car ce dernier ne permet pas d'écrire simultanément sur un même document.

        Enfin, il nous est demandé d'employer {\bf MS Project} comme outil de planification, que nous utiliserons donc pour tenir à jour le planning du projet.

	\subsection{Systèmes d'exploitation}
	Comme mentionné précédemment, nous nous focaliserons sur {\bf Windows} en raison de sa très grande présence dans les entreprises. Nous souhaitons cependant garder la porte ouverte à une compatibilité avec Linux et Macintosh, d'où le choix des langages de programmation décrits dans la section suivante.
	
    \subsection{Langages de programmation et environnements de développement intégrés}
        Nous comptons apporter quelques améliorations à ADTool, qui est écrit en {\bf Java}. Nous utiliserons donc {\bf IntelliJ IDEA} comme environnement de développement Java pour réaliser ces améliorations.
           
       	Pour le reste, l'emploi du {\bf C++} nous parait le meilleur compromis entre compatibilité et simplicité, tout en nous apportant une expérience importante avec ce langage. En effet, nous souhaitons éviter d'utiliser Java sur la totalité du projet afin d'élargir nos compétences en programmation. Le C++ étant très répandu en entreprise, nous souhaitons nous y confronter. Par conséquent, nous utiliserons {\bf Visual Studio} pour développer en C++, car c'est l'environnement de développement intégré le plus puissant et le plus utilisé actuellement pour ce langage.

    \subsection{Interfaces graphiques}
        L'interface graphique de notre suite est primordiale pour garantir une certaine ergonomie à l'utilisateur. Dans un souci de compatibilité avec Linux et Macintosh, notre choix s'est porté sur {\bf Qt} pour l'interface que présenteront les logiciels de notre suite --- autre que ADTool. 
        
        Pour ce dernier, nous considérons que le logiciel est déjà suffisamment simple et intuitif, et qu'y apporter des améliorations graphiques demanderait une charge de travail élevée pour une amélioration minime. Par conséquent, nous ne prévoyons pas de modification de son interface.