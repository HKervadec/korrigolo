\documentclass[a4paper]{article}

\usepackage[frenchb]{babel}
\usepackage[utf8]{inputenc}
\usepackage{amsmath}
\usepackage{graphicx}
\usepackage[colorinlistoftodos]{todonotes}
\usepackage[T1]{fontenc}
\usepackage{enumitem}
\usepackage{color}
\frenchbsetup{StandardLists=true}
\usepackage{pifont}

\begin{document}
    Cette nouvelle fonctionnalité, appelée « Éditeur de paramètres », prendrait donc en entrée les éléments suivants :
    \begin{itemize}[label=\ding{170},font=\color{magenta},parsep=0cm,itemsep=0cm]
        \item un arbre provenant du projet ;
        \item les paramètres intervenant dans la synthèse ;
        \item les opérations mathématiques appliquées ;
        \item le nom du paramètre de synthèse généré.
    \end{itemize}
\end{document}