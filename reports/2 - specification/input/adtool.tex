\section{ADTool}
	\label{sec:adtool}

	Présentation des tests et des limitations actuelles.

	\subsection{Couper copier coller}
	
	- Sélection d'un noeud (ça sélectionne aussi ses fils) : repérée par une coloration du noeud sélectionné
	- Clic droit sur le noeud sélectionné pour le copier/couper ainsi que ses fils
	- Pour coller on clique droit sur un noeud et cela rajoute l'arbre copié en tant que nouveau fils
	- On peut le faire entre différentes instances d'ADTool
	- Sanity check : faire attention au nom des labels

	\subsection{Amélioration XML}

	- Ajouter la possibilité d'éditer l'arbre directement en xml. Possibilité de masquer/afficher le bloc xml.
	- Ajouter noeud père dans l'éditeur texte 
	
	\subsection{CTRL-Z}

	- Annulation d'une ou plusieurs des modifications précédentes
	- Coder une pile d'états avec un curseur qui indique l'état courant
	- Chaque action entraîne la création d'un nouvel état
	
	\subsection{Paramètre de synthèse}
	- ajouter un paramètre de plus de disponible dans AdTool pour les fonctions de synthése.
		
	\subsection{Vue globale des paramètres}

	- Pour le moment, un onglet dans ADTool par paramètre (sur l'arbre, seul ce paramètre est apparent)	
	- Créer un nouvel onglet plus global avec tous les paramètres visibles sous les labels, chaque paramètre d'une couleur différente (légende explicative sur le coté)
	- Attention à ce que les paramètres ne débordent pas des noeuds, il faut éventuellement élargir les ronds