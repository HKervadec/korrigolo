\section{Rétrospective et conclusion}
	\paragraph{Rétrospective} Suite à la rédaction du rapport précédent (pré-étude), nous avions constaté que notre organisation de travail était à revoir. En effet, nous avions trop cloisonné la rédaction des différentes parties, et il nous manquait un superviseur chargé d'harmoniser le tout. Le résultat fut un rapport dont les parties ne s'enchaînaient pas toujours correctement, et avec des styles de rédaction très variables : cela manquait d'homogénéité.

	De plus, nous avions trop tardé à faire nos \og brainstorming \fg~ pour déterminer précisément ce que nous allions écrire, ce qui a grandement retardé la rédaction. Nous avions même dû supprimer une partie de notre travail.

	À partir de ces constats, nous avons décidé de démarrer ce deuxième rapport par une série de réunions. Cela nous a permis de commencer la rédaction en ayant une idée précise de ce que nous allions écrire. Cette fois, nous avons désigné un responsable d'harmonisation, Corentin {\sc Nicole}, qui a moins rédigé mais dont le rôle était de s'assurer de la cohérence globale du rapport tout au long de son avancement.

	\paragraph{Conclusion} Pour ce rapport, nous avons commencé par étudier le logiciel existant. Après avoir identifié ses limites, nous avons pu déterminer les principales modifications à y apporter. Il s'agit de l'ajout de paramètres de synthèse, d'un filtre et d'un optimiseur. Celles-ci seront implémentées dans notre logiciel, \glasir{}. Ces fonctionnalités sont totalement nouvelles et visent à améliorer les possibilités d'analyse des arbres pour l'expert en sécurité. 

	Nous avons ensuite décidé de modifier ADTool, qui jouera le rôle d'éditeur d'arbres dans notre architecture. Nous implémenterons des fonctionnalités ayant pour but de faciliter l'édition des arbres. Ainsi, la possibilité d'effectuer un copier/coller, une annulation d'action ou encore d'ouvrir plusieurs arbres à la fois nous a paru très utile à rajouter lors de nos tests.

	Ce rapport nous a permis de préciser le cahier des charges précédemment établi. De cette manière nous avons pu construire une ébauche de son architecture, avec les grands modules de notre projet et leurs interactions. Nous pouvons à présent travailler sur la planification précise du travail d'implémentation à effectuer.