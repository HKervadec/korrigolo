\section{Rétrospective et conclusion}
    \paragraph{Rétrospective} Suite à la rédaction du rapport précédent (pré-étude), nous avions constaté que notre organisation de travail était à revoir. En effet, nous avions trop cloisonné la rédaction des différentes parties, et il nous manquait un superviseur chargé d'harmoniser le tout. Le résultat fut un rapport dont les parties ne s'enchaînaient pas toujours correctement, et avec des styles de rédaction très variables : cela manquait d'homogénéité.

    De plus, nous avions trop tardé à faire nos \og brainstorming \fg{} pour déterminer précisément ce que nous allions écrire, ce qui a grandement retardé la rédaction. Nous avions même dû supprimer une partie de notre travail.

    À partir de ces constats, nous avons décidé de démarrer ce deuxième rapport par une série de réunions. Cela nous a permis de commencer la rédaction en ayant une idée précise de ce que nous allions écrire. Cette fois, nous avons désigné un responsable d'harmonisation, Corentin {\sc Nicole}, qui a moins rédigé mais dont le rôle était de s'assurer de la cohérence globale du rapport tout au long de son avancement.

    \paragraph{Conclusion} Après avoir présenté les limites du logiciel existant, ADtool, ce rapport a détaillé l'implémentation des principaux modules du logiciel \glasir{}. Celui-ci, développé essentiellement pour ce projet, permettra de dépasser les limitations d'ADTool grâce à trois fonctionnalités principales : l'éditeur de fonctions, l'optimiseur et le filtre. Ces nouvelles fonctionnalités visent à améliorer les possibilités d'analyse des arbres pour l'expert en sécurité. \glasir{} encapsulera ADTool et l'utilisera pour réaliser la création et l'édition des ADTrees.

    Dans un second temps, ce rapport a présenté les modifications qui seront apportées à ADTool dans le cadre de ce projet. Des fonctionnalités seront implémentées pour faciliter l'édition d'arbres. La possibilité d'effectuer un copier/coller, une annulation d'action ou encore d'ouvrir plusieurs arbres à la fois sont des exemples de fonctionnalités usuelles qui manquent à ADTool et qui y seront ajoutées.

    Ce rapport précise donc le cahier des charges précédemment établi. Il détaille l'architecture du logiciel \glasir{}, avec ses modules principaux et leurs interactions. La prochaine étape sera la planification précise du travail d'implémentation à effectuer.