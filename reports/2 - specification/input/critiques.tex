\section{Critiques}
	Après la rédaction du rapport précédent (pré-étude), nous avions constaté que notre organisation de travail était à revoir. 
	En effet, nous avions trop cloisonné la rédaction des différentes parties, avec personne chargé d'harmoniser le tout.
	Le résultat fut un rapport dont les parties s'enchainait mal, et avec un style de rédaction très variable.

	De plus, nous avions trop tardé à faire nos \og brainstorming \fg~ pour décider précisément ce que nous allions faire, ce qui a grandement retardé la rédaction, et nous avais même amené à jeter une partie de notre travail.

	A partir de ces constats, nous avons décidé de commencer par une série de réunion afin de commencer l'écriture du rapport en ayant une idée précise de ce que nous allions faire.
	Nous avons par ailleurs choisit un responsable d'harmonisation sur ce rapport, qui rédige moins mais s'assure de la cohérence globale du rapport.