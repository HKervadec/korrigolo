\section{Critiques}
	Après la rédaction du rapport précédent (pré-étude), nous avions constaté que notre organisation de travail était à revoir. 
	En effet, nous avions trop cloisonné la rédaction des différentes parties, et il nous manquait un superviseur chargé d'harmoniser le tout.
	Le résultat fut un rapport dont les parties ne s'enchaînaient pas toujours correctement, et avec des styles de rédaction très variables : cela manquait d'homogénéité.

	De plus, nous avions trop tardé à faire nos \og brainstorming \fg~ pour déterminer précisément ce que nous allions faire, ce qui a grandement retardé la rédaction. Nous avons même dû supprimer une partie de notre travail.

	À partir de ces constats, nous avons décidé de démarrer ce deuxième rapport par une série de réunions. Cela nous a permis de commencer la rédaction en ayant une idée précise de ce que nous allions écrire.
	Cette fois, nous avons désigné un responsable d'harmonisation, Corentin Nicole, qui rédige moins mais s'assure de la cohérence globale du rapport tout au long de son avancement.

% ABSENCE DE REFERENCES