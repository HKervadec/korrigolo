\section{Introduction}
	
	La sécurisation des systèmes est une problématique majeure de la société moderne. En ce sens, de nombreux outils ont été développés dans l'objectif d'identifier les risques et de les quantifier. C'est dans ce but que le concept d'arbres d'attaque et de défense (ADTrees) a vu le jour.
	
	Lors de la phase de pré-étude, nous avons pu comprendre l’intérêt pratique de la construction de ces arbres. Nous avons pris en main ADTool, logiciel libre développé pour l'implémentation de ces arbres sur support informatique. C'est alors que nous avons pu constater ses limites. En effet, dans un cas concret d'expertise en sécurité, les ADTrees produits sont de très grande taille. Dans ce cas, il est très difficile pour l'expert d'en extraire des informations pertinentes du premier coup d’œil. Or, aucun d'outil d'analyse n'est jusqu'alors inclus dans ADTool.

	L'objectif de notre projet sera donc la création d'un logiciel permettant de faciliter le travail de l'expert par l’intermédiaire de la simplification de l'analyse des ADTrees. Ce logiciel portera le nom de Glasir.

	Nous illustrerons les fonctionnalités de notre logiciel en prenant en exemple une situation précise. Il s'agira d'un expert en sécurité chargé par le Service des Transports en commun de l'Agglomération Rennaise (STAR) de déterminer les failles de leurs systèmes de paiement. 



% L'expert fait le travail humain : réflexion et blabla, mais le logiciel permet de simplifier "la vue de l'esprit"










