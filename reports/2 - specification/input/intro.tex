\section{Introduction}
	% Erreur de compilation sur les footnotes... si quelqu'un a une idée. SRY.
	% Faire une biblio ?
	
	La sécurisation des systèmes est une problématique majeure de la société moderne. En ce sens, de nombreuses méthodologie ont été développés. \footnote{\url{http://www.infres.enst.fr/people/leneutre/SECUR/INF941-Intro-securite-2011-12.pdf}}
	ont été développés dans l'objectif d'identifier les risques et de les quantifier. C'est dans ce but que le concept d'arbres d'attaque et de défense (ADTrees) a vu le jour.
	
	Lors de la phase de pré-étude, nous avons pu comprendre l’intérêt pratique de la construction des ADTrees. Leur utilisation permettent d'identifier de manière précise les différentes attaques possibles contre un système et de les valuer en termes de coût, de probabilité, etc. Nous avons pris en main ADTool\footnote{http://satoss.uni.lu/members/piotr/adtool/}, un logiciel libre développé pour l'implémentation de ces arbres sur support informatique. C'est alors que nous avons pu constater ses limites. En effet, dans un cas concret d'expertise en sécurite, le système doit faire face à une multitude d'attaques possibles et par conséquent l'ADTree qui les modélisera sera de très grande taille. Dans ce cas, il est très difficile pour l'expert d'en extraire des informations pertinentes du premier coup d’œil. Or, ADTool ne fournis pas d'outil permettant à l'utilisateur de simplifier sa lecture de l'arbre. 

	L'objectif de notre projet est donc la création d'un logiciel permettant de faciliter le travail de l'expert en sécurité. Pour cela, notre logiciel permettra à l'utilisateur de mettre en valeur des parties de l'ADTree préalablement produit. Par exemple, Il pourra obtenir différentes visualisation de son arbre en fonction du type d'attaquant. %repréciser encore et encore

	 Ce logiciel portera le nom de \glasir. Ce nom signifie \textit{rayon de lumière} en langue scandinave médiéval. Il s'agit du nom d'un arbre aux feuilles d'or dans la mythologie nordique. \footnote{Andy Orchard, Dictionary of Norse Myth and Legend, Cassell, 1997}

	Nous illustrerons les fonctionnalités de notre logiciel en prenant en exemple une situation précise. Il s'agira d'un expert en sécurité chargé par le Service des Transports en commun de l'Agglomération Rennaise (STAR) de déterminer les failles de leurs systèmes de paiement. 













