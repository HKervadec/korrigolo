\section{Introduction}
	La sécurisation des systèmes est une problématique majeure de la société moderne. En ce sens, de nombreuses méthodologies ont été développées~\cite{introSecurite,ADTreeKordy} dans le but d'identifier les risques et de les quantifier. C'est dans cet objectif que le concept d'arbres d'attaque et de défense (ADTrees) a vu le jour.

	Lors de la phase de pré-étude, nous avons pu comprendre l’intérêt pratique de la construction des ADTrees. Leur utilisation permet d'identifier de manière précise les différentes attaques possibles contre un système et de les valuer en termes de coût, de probabilité, etc. Nous avons pris en main ADTool~\cite{adtool_paper} (Attack-Defense Tree Tool), un logiciel libre développé pour l'implémentation de ces arbres sur support informatique. C'est alors que nous avons pu constater ses limites. En effet, dans un cas concret d'expertise en sécurité, le système doit faire face à une multitude d'attaques possibles et, par conséquent, l'ADTree qui les modélisera sera de très grande taille. Dans ce cas, il est très difficile pour l'expert d'en extraire des informations pertinentes au premier coup d’œil. Or, ADTool ne fournit pas d'outil permettant à l'utilisateur de simplifier l'analyse de l'arbre. 

	L'objectif de notre projet est donc la création d'un logiciel intégrant ADTool et permettant de faciliter le travail d'un expert en sécurité, en lui fournissant des outils pour analyser facilement ses ADTrees. Ce logiciel portera le nom de \glasir{}  (prononcé [\textipa{glaziK}]). Il s'agit du nom d'un arbre aux feuilles d'or dans la mythologie nordique~\cite{vikingCulture}.

	Ce rapport présente les spécifications fonctionnelles de \glasir{}. Tout d'abord, les limites d'ADTool seront abordées, afin de justifier l'intérêt de \glasir{}. Puis nous détaillerons les différentes fonctionnalités destinées à l'analyse des ADTrees. Enfin, des fonctionnalités supplémentaires offrant un meilleur confort seront également précisées. Ces spécifications seront faites en prenant en exemple une situation précise : celle du cas d'un expert en sécurité chargé par le Service des Transports en commun de l'Agglomération Rennaise (STAR) de déterminer les failles de leurs systèmes de paiement.