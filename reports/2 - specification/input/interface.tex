\section{Interface du logiciel}
L'interface utilisateur est la partie graphique de notre logiciel. C'est avec l'interface que l'utilisateur va interagir, elle doit donc être suffisamment ergonomique et intuitive. Elle communiquera avec les autres modules du logiciel et sert à faciliter les actions de l'utilisateur.
Celle-ci sera composée de trois parties principales.
	
	\subsection{ADTool}
La première sera l'intégration d'ADTool, située au milieu, et permettra la visualisation d'arbres, leur création ainsi que leur édition.
ADTool permettant déjà ces fonctionnalités, nous nous contenterons d'insérer ADTool dans notre logiciel, cette partie aura donc le même aspect.
	
		\subsubsection{Barre d'outils}
Les diverses améliorations auxquelles nous procéderons seront présentes dans la barre d'outils, également présente dans ADTool. Le menu \og File \fg contiendra les nouveaux imports et exports d'arbres. Le menu \og Edit \fg contient déjà les fonctions de couper/copier/coller, mais celles-ci ne s'appliquent pas aux arbres, nous rendrons donc cette action possible.
Le champ \og Domain \fg contiendra les nouvelles fonctions de valuation et leur description littéraire et mathématique, décrite de la même manière que les formules déjà présentes.

		\subsubsection{Édition d'arbres}
La partie consacrée à l'édition d'arbres, par fichier XML ou édition directe d'arbres, ne devrait pas connaître de changements majeurs. Des fonctionnalités seront ajoutées, comme le glisser/déposer, mais l'interface n'en sera pas affectée.

	\subsection{Arbres}
Ce qui concerne les arbres sera situé à gauche, comme indiqué sur la Figure. Cet onglet sera composé de deux parties.

		\subsubsection{Bibliothèque}
La bibliothèque sera visible en haut, et donnera accès à tous les arbres stockés précédemment. Nous implémenterons les fonctions essentielles permettant la bonne gestion de la bibliothèque, à savoir y ajouter des éléments, en retirer, ou réinitialiser la bibliothèque. L'importation d'une bibliothèque plus complète sera également possible.
Des catégories d'arbres seront créées afin de rendre les recherches plus aisées, et la navigation plus facile.

		\subsubsection{Liste d'arbres}
La liste sera située en dessous, et contiendra les arbres actuellement présents dans notre projet. Ceux créés dans le projet y seront ajoutés automatiquement, tandis que nous pourrons importer des arbres contenus dans la bibliothèque.

	\subsection{Fonctions}
Nous avons évoqué les fonctions de filtrage et de synthèse dans le cahier des charges. Elles seront disponibles directement dans l'interface, car nous considérons qu'elles jouent un rôle important dans notre logiciel. Celles-ci seront placées à la droite d'ADTool.

		\subsubsection{Fonction de synthèse}
Comme précisé dans le rapport précédent, les valuations sur les nœuds sont uniques, elles ne représentent qu'un seul attribut. Grâce à la fonction de synthèse, nous pourrons regrouper plusieurs attributs selon une formule que nous avons décidé, qui les représentera tous à la fois. L'éditeur de fonctions se composera des plusieurs champs. Nous donnerons un nom aux nouvelles synthèses, définirons à quelles variables elles s'appliquent et selon quelle formule. Une fois validée celle-ci sera stockée dans la liste de filtres à critères.

		\subsubsection{Filtre par critère}
Tous les critères seront regroupés ici. Nous pourrons ainsi décider de filtrer les arbres avec les différents critères choisis, afin de voir quelles attaques sont réalisables en fonction de ce que nous voulons. On pourra ainsi décider de ne laisser que les nœuds pour lesquels l'investissement nécessaire est inférieur à une certaine somme, pour ne garder que les solutions les moins coûteuses. Le filtrage sera réversible, on pourra donc retrouver l'arbre d'origine.
	
	\subsection{Guide}
Le guide sera facultatif, l'utilisateur, s'il est assez expérimenté ou tout simplement s'il le souhaite, pourra le désactiver. Afin de pouvoir le fermer simplement quand nous le souhaitons, il sera présent sous forme de bulle qui indiquera l'étape suivante dans la création de notre arbre.