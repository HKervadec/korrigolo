\section{\'Etude de l'existant}
	\label{sec:adtool}

	% Hierarchiser cette partie
	% Mettre en évidence fct qui manquent 

	ADTool permet de construire un ADTree sans difficulté. Le logiciel offre la possibilité de créer de nouveaux nœuds, d'éditer leurs labels et de leur ajouter des fils et des frères de façon simple et intuitive. Le choix des opérateurs entre ces nœuds --- conjonction ou disjonction ---  se fait sans effort, et la mise en place de défenses est également réalisable très facilement. % + figure ?
	
	\begin{figure}[h]
        \centering
        \includegraphics[width=1\textwidth]{figure/adtool_add_child.png}
        \caption{La création d'un arbre dans ADTool est simplifiée par des actions accessibles facilement.}
        \label{fig:arbre_exemple_1}
    \end{figure}
	
	Une fois l'arbre établi, il est possible d'ajouter des valuations à chaque feuille\footnote{Une feuille est un nœud n'ayant aucun fils.} de l'arbre selon un paramètre de base, tel que le coût minimal, la probabilité de succès, etc. (voir Section \ref{sec::ParamBase}). ADTool se charge ensuite de propager les valuations aux nœuds pères de façon récursive, jusqu'à la racine. % ajout de la liste des paramètres ?
	
	\begin{figure}[h]
        \centering
        \includegraphics[width=1\textwidth]{figure/adtool_add_values.png}
        \caption{L'ajout de valuations aux feuilles est simple. Ces valuations se propagent ensuite récursivement à tous les antécédents }
        \label{fig:arbre_exemple_1}
    \end{figure}
	
	Dans l'exemple de la {\sc Figure} \ref{fig:arbre_exemple_1}, chaque noeud possède une valuation selon le paramètre de base \og difficulté de réalisation \fg{}. Cette valuation peut prendre la valeur L, M, H ou E ((\emph{Low}, \emph{Medium}, \emph{High} ou \emph{Extrem} --- voir Section \ref{sec::ParamBase} pour plus de détails). En éditant la difficulté de réalisation de la feuille \og essayer les clés de chiffrage possibles \fg{}  à \emph{Medium}, la valuation du nœud père \og casser le chiffrage \fg{} est recalculée pour prendre la valeur \emph{Medium} : en effet, la difficulté de ce nœud conjonctif correspond à la difficulté la plus élevée parmi ses fils. Puis, à son tour, la valuation du père de ce nœud-père est recalculée à \emph{Middle}, car, étant un nœud disjonctif, sa difficulté de réalisation correspond à la difficulté la plus faible de ses fils.
	
	L'utilisateur n'a donc pas de calcul à faire lui-même pour obtenir la valuation de la racine de son arbre, c'est-à-dire la valuation de son objectif final. Cependant, l'analyse de l'arbre ainsi construit ne va pas plus loin : ADTool ne précise pas d'où provient la valuation d'un nœud. % mettre en évidence
	 Par conséquent, l'utilisateur n'a aucun moyen de déterminer le \og meilleur chemin \fg{} permettant d'atteindre la racine de l'arbre, par exemple. Il doit donc le déduire manuellement, en analysant comment les valuations se sont propagées. Sachant qu'un arbre atteint très rapidement une taille conséquente, cette opération peut prendre un temps très important, ce qui peut être un obstacle pour l'utilisateur.
	
	L'analyse de l'arbre est également restreinte par le fait que la valuation d'un nœud ne peut se faire que selon un seul paramètre. Il n'est pas possible de prendre en compte la \og difficulté \fg{} ET le \og temps nécessaire \fg{} à la réalisation d'un nœud, par exemple. Ceci est contraignant car il est difficile de séparer ces deux notions : par exemple, le nœud \og essayer les clés de chiffrage possibles \fg{} est très facile à effectuer (il suffit de taper les clés une-à-une jusqu'à trouver la bonne), mais le temps nécessaire pour effectivement essayer chaque clé est colossal. ADTool ne permet donc pas d'obtenir une vision plus complète de l'arbre par combinaison de plusieurs paramètres. % 2eme limitation : mettre en évidence.
	
	 De plus, lorsque l'utilisateur souhaite se focaliser sur une partie de son arbre, il ne peut que masquer tous les nœuds hiérarchiquement au-dessus (ou en dessous) d'un nœud spécifique, % 3ème limitation : mettre en évidence.
	  sans distinction. Il pourrait être intéressant d'avoir la possibilité de n'afficher que les nœuds dont la valuation correspond à certains critères (difficulté acceptable, temps raisonnable, etc.), et de masquer tous les autres. Ceci permettrait à l'utilisateur de s'affranchir des nœuds qui ne l'intéressent pas, ce qui faciliterait son analyse.
	 
	 
	 
