\section{Organisation}
	\subsection{Versions}
		Nous avons découpé le développement de Glasir en différentes versions, chacune incrémentale en fonctionnalités.
		Cela nous permettra de toujours avoir un produit fonctionnel, et facilitera les phases de tests.

		\begin{table}[h!]
			\begin{center}
			\begin{tabular}{|c|c|}
				\hline
				Version & Description\\
				\hline
				0.1 & Application qui lance ADTool\\
				\hline
				0.2 & Création d'un projet\\
				\hline
				0.3 & Paramètre de synthèse\\
				\hline
				0.4 & Filtre\\
				\hline
				0.5 & Optimiseur\\
				\hline
				0.6 & Bibliothèque de modèles\\
				\hline
				1.0 & Version pour la soutenance\\
				\hline
			\end{tabular}
			\end{center}
			\caption{Tableau regroupant nos versions}
		\end{table}

		Les améliorations sur ADTool seront réalisées en parallèle du développement de Glasir.

	\subsection{Planification}
		\begin{figure}
			\begin{center}
				\includegraphics[width=0.5\textwidth]{figure/planif.jpg}
			\end{center}
			\caption{Planif MS Project}
			\label{fig:planif}
		\end{figure}


	\subsection{Répartition des tâches}
