\section{Organisation}
	 Cette planification sera le sujet d'un prochain rapport en décembre, mais nous pouvons déjà en donner une ébauche. Celle-ci sera entre autres illustrée par une chronologie MS Project.

	\subsection{Versions}
		Nous avons découpé le développement de \glasir en différentes versions, chacune incrémentale en fonctionnalités. Cela nous permettra de toujours avoir un produit fonctionnel, et de faciliter les phases de tests.

		\begin{table}[h!]
			\begin{center}
			\begin{tabular}{|c|c|}
				\hline
				Version & Description\\
				\hline
				0.1 & Application qui lance ADTool\\
				\hline
				0.2 & Création d'un projet\\
				\hline
				0.3 & Paramètre de synthèse\\
				\hline
				0.4 & Filtre\\
				\hline
				0.5 & Optimiseur\\
				\hline
				0.6 & Bibliothèque de modèles\\
				\hline
				1.0 & Version pour la soutenance\\
				\hline
			\end{tabular}
			\end{center}
			\caption{Tableau énumérant les différentes versions de \glasir.}
		\end{table} % ajouter release date ? 

		Toutes les améliorations sur ADTool seront réalisées au fur et à mesure, en parallèle du développement de \glasir.

	\subsection{Planification}
		Une ébauche de la planification se trouve à la {\sc{Figure}} \ref{fig:planif}. Cette dernière donne un aperçu du temps de développement qui sera consacré à chaque version de \glasir. 

		\begin{landscape}
			\begin{figure}
				\centering
				\includegraphics[height=0.60\textwidth]{figure/planification.png}
				\caption{Planification modélisée sous MS Project.}
				\label{fig:planif}
			\end{figure}
		\end{landscape}

	\subsection{Répartition des tâches}
		Au second semestre, seuls resteront Pierre-Marie {\sc Airiau}, Valentin {\sc Esmieu} et Maud {\sc Leray} à travailler sur ce projet, étant donné que le reste du groupe part étudier à l'étranger. Par conséquent, nous comptons séparer le travail ainsi :
		\begin{itemize} 
			\item l'un d'entre nous développera les fonctionnalités d'analyse de \glasir à proprement parler ;
			\item le deuxième se tournera plutôt vers l'intégration d'ADTool dans \glasir ;
			\item le troisième apportera les modifications à ADTool.
		\end{itemize}

		Cette répartition sera probablement affectée par les semaines de partiels qui vont différer entre nous trois sur la fin de l'année. 