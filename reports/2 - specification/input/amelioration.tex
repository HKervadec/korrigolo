\section{Amélioration}


	\subsection{Bibliothèque de modèles}

		\begin{itemize}
			\item Le logiciel posséde une bibliothéque par défault
			\item la création d'un projet crée une nouvelle bibliothéque, copie de la bibliothéque par défault
			\item la bibliothéque est contenue dans le fichier projet de l'utilisateur. 
			\item Permet un accès facile à tous les arbres stockés
			\item bouton “actualiser” en haut pour mettre à jour la liste des fonctions.
			\item Regroupe les arbres par catégorie (par catégorie d’attaquant, de type d’attaque…), vue en arborescense. 	donner des types aux arbres stockés ?
			\item Permet d’y ajouter des éléments (géré automatiquement lors de la création dans un projet ?) Si non, bouton “ajouter” pour l’arbre courant. 
			\item Retirer des éléments : permet de retirer un élément et mettre à jour la bibliothèque, utile en cas d’arbre faux, ou considéré comme pas assez général pour servir de modèle. (cohérence avec l’ajout d’arbres)
			\item Pouvoir rechercher un arbre dans les modèles généraux (aussi rôle du guide) mais guide facultatif. Série de questions pour pouvoir sélectionner des modèles.
			\item Possibilité d'ajouter des arbres dans la bibliothéque par défault.\ldots
		\end{itemize}
		

	\subsection{Liste d'arbres}

		\begin{itemize}
			\item Liste d'arbres générés par l'utilisateur puis sauvegardés sous différentes catégories/sous-catégories (arborescence)
			\item Au départ, une seule catégorie défaut
			\item  Création des catégories + possibilité d'affiner en sous-catégories
			\item Suppression d'une catégorie supprime tous les arbres compris dedans
			\item bouton “actualiser” en haut pour mettre à jour la liste des fonctions.
			\item Glisser/déposer les arbres pour les déplacer entre les catégories
			\item Possibilité de supprimer les arbres (renommer aussi ?)
			\item Double-clic pour ouvrir les arbres dans une fenêtre ADTool.\ldots
		\end{itemize}
		
	\subsection{Éditeur d'arbres}

		Attention: on ne parlera ici que de l'intéraction entre l'éditeur et les différents modules, pas de l'éditeur en lui même.
		Voir la section \ref{sec:adtool} pour voir nos améliorations.

		\begin{itemize}
			\item  Utilisation d'ADTool imbriqué dans l'IHM
			\item Une instance d'ADTool par arbre (possibilité d'ouvrir plusieurs onglets) : modèles, filtres, etc
			\item Chaque modification dans ADTool entraine une mise à jour de SAD
			\item Sauvegarde des fichiers par ADTool : ne pas laisser le choix du fichier, laisser seulement le fichier projet (en gros, l'arborescence d'arbres)\ldots
		\end{itemize}


	\subsection{Couper copier coller}
	
	ADTool ne permet pas ces fonctions, qui pourraient pourtant s'avérer précieuses dans la création des arbres. En effet, dans le cas de l'oubli d'un niveau, on peut souhaiter copier ou couper les fils d'un noeud pour les coller dans un second plus haut dans la généalogie.
	En conséquence, la sélection d'un noeud entrainera la sélection de ses fils, afin de pouvoir effectuer des transferts rapidement. Nous avons considéré que la sélection de seulement certains noeuds fils n'apportait pas d'amélioration.
	- Sélection d'un noeud (ça sélectionne aussi ses fils) : repérée par une coloration du noeud sélectionné
	- Clic droit sur le noeud sélectionné pour le copier/couper ainsi que ses fils
	- Pour coller on clique droit sur un noeud et cela rajoute l'arbre copié en tant que nouveau fils
	- On peut le faire entre différentes instances d'ADTool
	- Sanity check : faire attention au nom des labels

	\subsection{Amélioration XML}

	- Ajouter la possibilité d'éditer l'arbre directement en xml. Possibilité de masquer/afficher le bloc xml.
	- Ajouter noeud père dans l'éditeur texte 
	
	\subsection{CTRL-Z}

	- Annulation d'une ou plusieurs des modifications précédentes
	- Coder une pile d'états avec un curseur qui indique l'état courant
	- Chaque action entraîne la création d'un nouvel état
	
	\subsection{Paramètre de synthèse}
	- ajouter un paramètre de plus de disponible dans AdTool pour les fonctions de synthése.
		
	\subsection{Vue globale des paramètres}

	- Pour le moment, un onglet dans ADTool par paramètre (sur l'arbre, seul ce paramètre est apparent)	
	- Créer un nouvel onglet plus global avec tous les paramètres visibles sous les labels, chaque paramètre d'une couleur différente (légende explicative sur le coté)
	- Attention à ce que les paramètres ne débordent pas des noeuds, il faut éventuellement élargir les ronds