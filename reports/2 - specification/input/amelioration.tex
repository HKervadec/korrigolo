\section{Amélioration}

	Dans le but de simplifier l'expérience de l'utilisateur, nous avons souhaité ajouter certaines fonctionnalités à notre logiciel. Si celles-ci n'apportent pas d'amélioration en terme d'analyse des arbres, elles permettent de faciliter grandement leur modification. Nous distinguerons deux types de modifications : celles qui simplifient l'édition de l'arbre par l’intermédiaire d'AdTool, qui se traduiront par une modification de ce logiciel, et celles intégrées à \textbf{Glasir} qui simplifieront la navigation entre les différents arbres (qu'ils soient réalisés par l'utilisateur lui-même ou importés comme modèles). 

	\subsection{Liste d'arbres}.

	Dans sa version actuelle, ADTool ne permet d'ouvrir qu'un seul arbre à la fois : cette limitation est particulièrement handicapante. Par exemple, l'utilisateur ne peut pas ouvrir deux versions différentes d'un même arbre pour les comparer. Notre logiciel rendra donc possible l'ouverture simultanée de plusieurs onglets, chacun possédant un arbre différent. 

	Cette fonctionnalité induit la possibilité pour l'utilisateur de manipuler un grand nombre d'arbres en simultané dans le cadre d'un même projet. Le module \emph{Liste d'arbres} fournira à l'utilisateur une présentation de la liste des arbres du projet sous forme d'arborescence. Il s'agira d'un dock disponible sur le côté du logiciel. 

	L'utilisateur aura aussi la possibilité de créer des catégories et des sous-catégories d'arbres pour hiérarchiser son projet. La suppression, la création et le renommage d'un arbre seront réalisables directement depuis ce module.
	
	\subsection{Bibliothèque de modèles}

	Il n'est pas toujours aisé pour l'utilisateur de créer un arbre de but en blanc, en partant de rien. C'est pourquoi \emph{Glasir} fournira à l'utilisateur une bibliothèque de modèles, chacun pouvant servir de base pour de nouveaux arbres. Cette bibliothèque contiendra un ensemble d'arbres génériques, que l'utilisateur pourra personnaliser et compléter pour créer un nouvel arbre. Il pourra aussi l'utiliser tel quel s'il lui convient. La création d'un nouveau projet entraînera une copie de la bibliothèque de modèles de base, qui pourra ensuite être modifiée par l'utilisateur. Une bibliothèque de modèles est donc propre à chaque projet. Par exemple, si l'expert crée un arbre qu'il juge utile de réexploiter plusieurs fois dans le projet, il lui sera possible de l'intégrer à la bibliothèque de modèles.

	\subsection{Éditeur d'arbres}

	L'édition des arbres dans \emph{Glasir} se fera exclusivement par l'intermédiaire d'ADtool. L'ouverture d'un arbre par l'utilisateur provoque le lancement d'une instance d'ADTool contenant cet arbre. ADTool n'offrant pas la possibilité d'ouvrir plusieurs arbres à la fois, chaque instance (une par arbre) est imbriquée dans l'IHM du logiciel sous la forme d'onglet. Pour rendre possible l'interaction entre \emph{Glasir} et ADtool, nous apporterons quelques modifications à ce dernier, qui seront explicitées par la suite.

\subsection{Couper/copier/coller}
	
	Actuellement, ADTool ne permet pas l'utilisation des fonctions couper/copier/coller qui pourraient pourtant s'avérer pratiques lors de la création d'un arbre. Par exemple, dans le cas de l'oubli d'un nœud père, l'instauration de ces fonctionnalités permettra de déplacer facilement les fils concernés de l'ancien nœud père vers le nouveau.\\
	En conséquence, la sélection d'un nœud entraînera la sélection de ses fils, afin de pouvoir effectuer des transferts rapidement. Ceci sera automatique, car nous avons considéré que la sélection partielle des nœuds fils ne présentait pas vraiment d'utilité. Le système de raccourcis clavier classique ne sera probablement pas repris, au profit d'une gestion à la souris et d'un affichage spécial des nœuds copiés (coloration particulière, par exemple).\\

	Cependant, cette fonctionnalité entraîne des questions de cohérence. En effet, des nœuds de même label ne pourront pas être présents, il faudra donc gérer ces cas de figure. L'édition XML abordée ci-dessous sera le moyen le plus simple de résoudre ces éventuels conflits.

	\subsection{Amélioration du codage des arbres}

	Tel qu'il existe aujourd'hui, ADTool affiche dans son interface un section nommée \emph{ADTerm Edit}. Celle-ci affiche une représentation de l'arbre sous la forme d'un texte utilisant un langage propre au logiciel. Lors de la modification de l'arbre dans l'éditeur graphique, AdTool modifie le code en temps réel. On peut ainsi modifier facilement les labels des nœuds présents, ou changer un opérateur directement depuis \emph{ADTerm Edit} puis valider afin d'afficher le résultat sous forme d'arbre. 

	Mais le langage qui permet de décrire les arbres ne contient pas le nom des autres nœuds que les feuilles. Sur la figure \ref{fig:int_adTool}, on peut constater que le code indique le nom des deux feuilles \textit{Acheter le matériel nécessaire} et \textit{Essayer les clés de chiffrage}, que la conjonction est bien précisé par l'opérateur \textit{ap} mais qu'aucune référence n'est faite au label du nœud parent (\textit{Casser le chiffrage}).

	Pour remédier à ce problème, nous créerons une nouvelle grammaire corrigeant ce défaut.

	%- Ajouter la possibilité d'éditer l'arbre directement en xml. Possibilité de masquer/afficher le bloc xml.
	%- Ajouter nœud père dans l'éditeur texte 


	% ATTENTION  !! CHANGER CRIPTAGE EN CHIFFFRAGE !!!
	\begin{figure}
		\centering
		\includegraphics[width=0.5\textwidth]{figure/interface_adtool.png}
		\caption{L'interface d'AdTool.}
		\label{fig:int_adTool}
	\end{figure}
	
	\subsection{CTRL-Z}
	Pour le moment, effectuer une action est irréversible. Un renommage ou l'ajout d'une feuille est rapide à effectuer, mais supprimer par erreur un nœud peut entraîner un travail énorme. La possibilité de faire un retour à l'état précédent de l'arbre éviterait d'avoir à recommencer ce travail de construction fastidieux. Nous souhaitons créer au moins une sauvegarde de l'état précédent, afin de pouvoir annuler la dernière modification. Si possible, l'implémentation d'une pile circulaire contenant les N derniers états avec un curseur pointant sur l'état courant, permettra de revenir en arrière sans contraintes.
	%- Annulation d'une ou plusieurs des modifications précédentes
	%- Coder une pile d'états avec un curseur qui indique l'état courant
	%- Chaque action entraîne la création d'un nouvel état

	\subsection{Vue globale des paramètres}
	Dans l'état actuel d'ADTool, un seul paramètre est affiché, même si plusieurs paramètres servent à valuer le nœud. En effet, les paramètres ont un onglet propre, et un seul est effectivement affiché. Nous souhaitons créer un onglet général, regroupant tous les paramètres appliqués, afin d'avoir une vision plus globale de l'avancée de notre travail, et ainsi faciliter la création des fonctions de synthèse.
	Pour une meilleure lisibilité, les différents paramètres auront chacun une couleur différente, accompagnée d'une légende résumant à quoi elles correspondent.
	Nous devrons également gérer la taille des nœuds, pour que les paramètres ne dépassent pas de la bulle. Un tel système est déjà présent dans ADTool afin de gérer les labels, il suffira de l'étendre à l'affichage des paramètres.
 	%Pour le moment, un onglet dans ADTool par paramètre (sur l'arbre, seul ce paramètre est apparent)	
	%- Créer un nouvel onglet plus global avec tous les paramètres visibles sous les labels, chaque paramètre d'une couleur différente (légende %explicative sur le coté)
	%- Attention à ce que les paramètres ne débordent pas des nœuds, il faut éventuellement élargir les ronds
