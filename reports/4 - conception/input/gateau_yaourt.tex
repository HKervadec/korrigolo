\documentclass[a4paper]{article}

\usepackage[french]{babel}
\usepackage[utf8]{inputenc}
\usepackage{lmodern}
\usepackage{amsmath}
\usepackage{graphicx}
\usepackage[colorinlistoftodos]{todonotes}
\usepackage[T1]{fontenc}
\usepackage{enumitem}
\usepackage{color}
\usepackage{pifont}

\begin{document}
{\bf Gâteau au yaourt (pour les nuls)} :

\vspace{10mm}

Préchauffer le four à 180, puis mettre dans un saladier :

\vspace{2mm}

\begin{itemize}[label=\ding{170} \Large,font=\color{magenta},parsep=0cm,itemsep=0cm]
        \item un yaourt dont on utilise ensuite le pot comme mesure ;
        \item 2 oeufs ;
        \item un peu d'huile neutre ;
        \item un pot de sucre ;
        \item un sachet de sucre vanillé (facultatif).
\end{itemize}

\vspace{2mm}

Bien mélanger, puis ajouter petit à petit :

\vspace{2mm}

\begin{itemize}[label=\ding{170} \Large,font=\color{magenta},parsep=0cm,itemsep=0cm]
        \item 3 pots de farine ;
        \item un sachet de levure ;
        \item garniture si envie soudaine (chocolat, pommes, etc).
   \end{itemize} 

\vspace{2mm}

Verser la pâte dans un moule à gâteau beurré, puis enfourner environ 25 min (vérifier que la pointe d'un couteau ressort bien sèche).

\end{document}