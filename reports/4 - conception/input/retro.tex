\section{Rétrospective}
    \label{sec:retro}

    Une ébauche du cahier des charges de la conception de Glasir a été fournie en novembre, dans le rapport de pré-étude~\cite{pre_etude}. Cette première version du projet a quelque peu évolué suite aux réunions de groupe qui ont suivi, et plus encore lors de la rédaction du présent rapport.

    Par exemple, le cahier des charges initial prévoyait l'implémentation d'un \og Guide \fg{} dans Glasir, afin d'aider un éventuel utilisateur n'étant pas familier avec les ADTrees. En effet, une limitation non-négligeable de Glasir est son accessibilité limitée : un expert en sécurité ne pourra l'utiliser que s'il dispose de connaissances de base sur la théorie des ADTrees. Il doit comprendre leur représentation, et leurs valuations, afin de pouvoir ensuite les interpréter et les analyser. Prévu sous forme de journal de quêtes, le guide aurait permis d'expliquer chaque étape d'utilisation de Glasir, de la création d'un ADTree jusqu'à son analyse. Cependant, après discussion de cette idée avec nos encadrants, il est apparu que ceci relevait plus du domaine de la recherche que de celui d'un projet étudiant, et que le travail engendré serait trop conséquent. Le guide a donc été retiré de la planification.

    Il était également indiqué dans le rapport de pré-étude que les technologies utilisées seraient le C++ pour Glasir, et Qt pour son interface graphique. Finalement, vu le temps nécessaire à un apprentissage poussé de ces technologies, et suite au départ de la moitié du groupe en semestre d'étude à l'étranger, il a été décidé que Glasir serait codé en C\# et l'interface graphique en WPF, le tout sous Visual Studio. Toutefois, ADTool restera bien en Java, et sera amélioré dans ce langage. Le choix de ces technologies pose plus de problèmes de compatibilité multi-plateformes, mais nous préférons donner la priorité au bon fonctionnement du logiciel sous Windows plutôt que de fournir une version hasardeuse qui fonctionnerait sur toutes les plateformes. Par conséquent, nous développerons l'application pour Windows, sur lequel les tests seront effectués ; d'autres tests sur d'autres systèmes d'exploitation tels que GNU/Linux ou Mac OS ne sont pas prévus.