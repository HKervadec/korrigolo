\section{Fonctionnalités supplémentaires d'ADTool}
    \label{sec:ADTool}
    Cette section décrit la conception des fonctionnalités qui vont être ajoutées à ADTool afin de le rendre plus ergonomique.
    
    \subsection{Annulation des dernières actions}
    	La {\sc Figure}~{\ref{fig:ctrlz}} illustre l'agencement des classes permettant de réaliser la fonctionnalité d'annulation d'ADTool. Tout d'abord, chaque action annulable possède sa propre classe définissant l'action à exécuter, ainsi que l'action opposée permettant de l'annuler. Par exemple, les actions permettant d'ajouter un nœud-fils, de changer le label d'un nœud ou encore d'ajouter un paramètre possèdent leurs classes respectives \emph{AddChildEdit}, \emph{ChangeLabelEdit} et \emph{AddDomainEdit}. Ces dernières travaillent toutes sur un ADTree, représenté ici par une classe \emph{ADTree} pour simplifier le diagramme. Ces classes implémentent également \emph{UndoableEdit}, l'interface générique d'une action annulable. Enfin, un gestionnaire d'actions \emph{HistoryManager} stocke les actions effectuées dans un attribut, tout en disposant de méthodes permettant d'annuler ces actions dans l'ordre inverse de celui de leur réalisation.
    	
    	\begin{figure}[H]
	        \centering
	        \includegraphics[height=0.8\textwidth]{figure/ctrlz.png}
	        \caption{Diagramme de classes de l'annulation des dernières actions.}
	        \label{fig:ctrlz}
	    \end{figure}
    
    \subsection{Couper-copier-coller}
		La {\sc Figure}~{\ref{fig:copiercoller}} illustre le diagramme de classes permettant de réaliser le couper-copier-coller d'ADTool. Les ADTrees, représentés par la classe \emph{ADTree}, implémentent \emph{Transferable}, l'interface permettant à un élément d'être coupé, copié ou collé. ADTool, schématisé par la classe \emph{ADTool}, se voit doté d'un \emph{ADTreeTransfertHandler}, une classe permettant de réaliser les actions couper-copier-coller en elles-mêmes. \emph{ADTreeTransfertHandler} implémente \emph{TransfertHandler}, l'interface qui permet de gérer un \og presse-papier \fg{}. Ce dernier est représenté par la classe \emph{Clipboard}, et contient l'élément coupé ou copié, c'est-à-dire un ADTree.
    	
    	\begin{figure}[H]
	        \centering
	        \includegraphics[height=0.6\textwidth]{figure/copiercoller.png}
	        \caption{Diagramme de classes du couper-copier-coller.}
	        \label{fig:copiercoller}
	    \end{figure}
	    
	\subsection{Vue globale des paramètres}
		Cette vue globale permet de présenter un condensé de l'ensemble des paramètres présents sur un ADTree au sein d'un seul onglet. Pour ce faire, il est nécessaire d'appliquer quelques changement d'affichage...
		
		Section à continuer !
	
	\subsection{Amélioration de la représentation textuelle}
		Pour rappel, ADTool dispose d'une zone dans laquelle l'ADTree courant est représenté sous forme textuelle. Cette représentation ne contient que les labels des nœuds n'ayant aucun fils : tous les autres nœuds sont représentés par \emph{op} ou \emph{ap}, selon que ledit nœud soit conjonctif ou disjonctif. Pour améliorer la lisibilité de cette zone textuelle, la représentation doit inclure les labels de ces nœuds. Ceci est réalisé par la modification de deux fichiers d'ADTool :
		\begin{itemize}
		\item \emph{ADTNode.java}, la classe représentant un nœud au sein d'un ADTree ;
		\item \emph{adtparser.jit}, le fichier générant la zone textuelle en elle-même.
		\end{itemize}
		
		À travers ces deux fichiers, il est possible de faire apparaître les labels des nœuds possédant des fils au sein de la zone textuelle d'ADTool.
		
	  