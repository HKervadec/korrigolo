\section{Limites de Glasir}
    \label{sec:limites}

    Lors de la phase de réflexion sur les fonctionnalités du logiciel, certaines sont apparues comme trop compliquées à réaliser par rapport au temps imparti pour le projet. Malgré les qualités intéressantes qu'elles pourraient apporter à Glasir, elles ne sont pas fondamentales et n'ont donc pas été prises en compte dans la planification. Ces limitations sont présentées dans cette section.

    \paragraph{Pré-requis nécessaires}

    Une limitation non négligeable de Glasir est son accessibilité limitée. En effet, un expert en sécurité ne pourra l'utiliser que s'il dispose de connaissances de base sur la théorie des ADTrees. Il doit comprendre leur représentation, et leurs valuations, afin de pouvoir ensuite les interpréter et les analyser. Pour contourner cette difficulté, une idée de \og guide \fg{} avait été évoquée dans le rapport de pré-étude~\cite{pre_etude}, pour aider un éventuel utilisateur n'étant pas familier avec les ADTrees. Cependant, après avoir discuté de cette idée avec nos encadrants, il est apparu que cela relevait plus du domaine de la recherche que de celui d'un projet étudiant, et que le travail engendré serait trop conséquent.

    \paragraph{Systèmes d'exploitation}

    ADTool est codé en Java, et Glasir sera majoritairement en C\#. Ces deux langages de programmation permettent en théorie une utilisation multi-plateformes, ce qui permettrait de toucher un panel d'utilisateurs plus large. Cependant, comme cela était expliqué dans le rapport de pré-étude~\cite{pre_etude}, nous développerons l'application pour Windows, sur lequel les tests seront effectués. D'autres tests sur d'autres systèmes d'exploitation (tels que GNU/Linux ou Mac OS) ne sont pas prévus dans la planification du projet, principalement pour des raisons de temps.

    \paragraph{Annulation de l'action précédente}

    Il a été prévu d'ajouter à ADTool un certain nombre de fonctionnalités, dont l'annulation d'une action. Cette amélioration était présentée dans le rapport de spécifications fonctionnelles~\cite{spec_fonc}, où il était précisé que nous souhaitions créer au moins une sauvegarde de l'état précédent, afin de pouvoir annuler la dernière modification effectuée. L'implémentation d'une pile circulaire permettant de revenir plus loin en arrière n'est pour le moment pas planifiée, car nous avons préféré donner la priorité aux fonctionnalités d'analyse principales de Glasir.