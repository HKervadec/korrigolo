\section{Conclusion}
    \label{sec:conclusion}

    Ce rapport a débuté par un aperçu global des possibilités offertes par Glasir suivi d'une présentation d'un prototype d'interface. Cette première partie a permis de mieux comprendre la conception générale du logiciel. Les trois parties suivantes, contenant différents diagrammes de classes et de séquence, ont donné plus de détails sur la structure de Glasir et sur les savoir-faire utilisés. Ainsi, l'implémentation des fonctionnalités principales de Glasir mais aussi les modifications apportées à ADTool ont été explicitées. Nous avons ensuite fait le point sur la progression du projet par le biais d'une rétrospective, qui a été l'occasion de justifier quelques petits changements de programme. Ces derniers ont alors été pris en compte dans notre organisation, ce qui nous permet donc de continuer l'implémentation de Glasir que nous avons amorcée lors de la rédaction de ce rapport. La livraison de la version intermédiaire 0.1 est en effet prévue pour le 20 mars 2015.