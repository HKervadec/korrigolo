\section{Organisation et planning}
    \label{sec:orga}

    L'organisation de la réalisation de Glasir a déjà été détaillée lors du rapport de planification~\cite{planif}. Il y était précisé que la méthode de gestion de projet suivie serait celle du SCRUM, qui s'apparentait le plus à notre façon de travailler. Il était aussi indiqué que trois versions de Glasir seront livrées : deux intermédiaires (0.1 et 0.2) et une finale (1.0), dont le développemet a été partitionné en tâches unitaires. Ces deux faits sont toujours d'actualité, et constitueront le fil conducteur du projet à partir de maintenant. 

    Cependant, le début de l'implémentation de la version 0.1 ne devait commencer qu'à la fin de la rédaction du présent rapport de conception. En réalité, nous avons pu en parallèle du rapport commencer à coder Glasir, entre autres en créant un prototype d'interface ({\sc Figure}~\ref{fig:interface}) sous Visual Studio. Nous nous sommes également intéressés au code source d'ADTool, et avons pu rencontrer son développeur Piotr {\sc Kordy} afin de lui poser nos questions. La prochaine étape est donc, comme prévu, de livrer la version 0.1 avec son Éditeur de fonctions ainsi que les modifications prévues pour ADTool (vue globale des paramètres et amélioration de la représentation textuelle des ADTrees).