\section{Rétrospective}
\label{sec:retro}

\subsection{Remarques générales}
\label{ssec:rq_gen}

Ce projet de 4\ieme{} année Informatique fut une expérience très enrichissante. Tout d'abord, il nous a permis de beaucoup progresser en matière de programmation, dans les technologies utilisées : principalement C\#, Java et WPF dans notre cas. Ensuite, nous avons été confrontés aux problématiques du travail en équipe : présence d'un coordinateur, répartition des tâches, utilisation des outils de gestion de versions (Git dans notre cas), etc. Enfin, nous avons pu gérer de A à Z un projet d'une assez grande envergure, ce qui a nécessité plusieurs phases de préparation avant de pouvoir démarrer : pré-étude~\cite{pre_etude}, spécifications fonctionnelles~\cite{spec_fonc}, planification~\cite{planif} puis finalement conception~\cite{conception}. C'est sur l'avant-dernière de ces phases, celle de planification, que nous allons à présent revenir pour présenter un bilan de notre gestion de projet dans la {\sc sous-section}~\ref{ssec:gestionProjet}.

\subsection{Réflexions sur la gestion de projet}
\label{ssec:gestionProjet}

Cette sous-section n'est qu'un rapide bilan concernant notre gestion de projet, puisqu'un rapport de bilan de planification~\cite{bilanPlanif} a été rédigé en parallèle du présent rapport et comporte plus de détails à ce sujet.

Les quelques écarts constatés à la {\sc section}~\ref{sec:rect} entre les fonctionnalités promises et celles développées s'expliquent par certains défauts que nous avons identifiés dans notre gestion de projet, et qui sont les suivants : 

\begin{itemize}
\item une phase de pré-étude insuffisante au niveau technique ;
\item un cahier des charges trop vague ;
\item une méthodologie de tests trop approximative.
\end{itemize}

Ces points à améliorer sont développés dans le rapport bilan de planification~\cite{bilanPlanif}, c'est pourquoi nous ne nous attardons pas dessus ici. En revanche, ils donnent déjà des idées de pistes à suivre pour améliorer la gestion de nos projets à venir.

Cependant, la planification initialement prévue dans le rapport de planification~\cite{planif} a globalement été suivie. Les délais fixés ont été respectés, et les fonctionnalités annoncées ont presque toutes été implémentées, et ce de manière fonctionnelle. Un récapitulatif de cette tenue des objectifs a d'ailleurs été présenté dans la {\sc sous-section}~\ref{ssec:bilan} de ce rapport, et justifié par la suite. 